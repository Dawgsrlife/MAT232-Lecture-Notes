\documentclass{article}
\usepackage{amsthm,amsmath,amsfonts,lipsum}
\usepackage[T1]{fontenc}
\usepackage{beramono}
\usepackage{listings}
\usepackage{fontawesome5}
\usepackage{adjustbox}
\usepackage{mathabx}
\usepackage{thmtools}
\usepackage{import}
\usepackage{graphicx}
\usepackage{setspace}
\usepackage{geometry}
\usepackage{physics}
\usepackage{float}
\usepackage[english]{babel}
\usepackage{framed}
\usepackage[dvipsnames,x11names]{xcolor}
\usepackage{tcolorbox}
\usepackage{fancyhdr}
\usepackage{hyperref}
\usepackage{booktabs}
\usepackage{enumitem}
\usepackage{cancel}
\usepackage{background}
\usepackage{units}

% Configuring the background
\backgroundsetup{
  scale=1, % Optional, scale if needed
  color=black, % Optional, set the image color, can be omitted
  opacity=0.18, % Optional, adjust opacity for watermark effect
  angle=0,
  position=current page.center, % Center the image on the page
  contents={\includegraphics[width=1.75\paperwidth, height=1.75\paperheight, keepaspectratio]{ninym_ralei_leaf (watermarked by AlexanderTheMango)}} % Keeps aspect ratio and scales to fill the page
}

% Colours
\definecolor{darkgreen}{rgb}{0.0, 0.5, 0.0}

% Define custom tcolorbox styles for notes
\tcbuselibrary{skins, breakable}
\newtcolorbox{definitionbox}{colframe=RoyalBlue, colback=blue!5!white, title=Definition}
\newtcolorbox{examplebox}{colframe=ForestGreen, colback=green!5!white, title=Example}
\newtcolorbox{notebox}{colframe=RedOrange, colback=orange!5!white, title=Note}
\newtcolorbox{theorembox}{colframe=RoyalPurple, colback=purple!5!white, title=Theorem}

\newtcolorbox{propositionbox}{colframe=Goldenrod, colback=yellow!10!white, title=Proposition}
\newtcolorbox{remarkbox}{colframe=MidnightBlue, colback=blue!10!white, title=Remark}
\newtcolorbox{corollarybox}{colframe=OliveGreen, colback=green!10!white, title=Corollary}
\newtcolorbox{warningbox}{colframe=Crimson, colback=red!5!white, title=Warning}
\newtcolorbox{proofbox}{colframe=Black, colback=gray!10!white, title=Proof}
\newtcolorbox{questionbox}{colframe=Teal, colback=teal!10!white, title=Question}
\newtcolorbox{tipbox}{colframe=Goldenrod, colback=yellow!10!white, title=Tip}
\newtcolorbox{exercisebox}{colframe=darkgreen, colback=green!5!white, title=Exercise}
\newtcolorbox{solutionbox}{colframe=DodgerBlue4, colback=blue!5!white, title=Solution}
\newtcolorbox{algorithmbox}{colframe=Navy, colback=blue!10!white, title=Algorithm}
\newtcolorbox{conceptbox}{colframe=Chocolate, colback=brown!10!white, title=Concept}
\newtcolorbox{historicalbox}{colframe=Firebrick, colback=red!10!white, title=Historical Note}
\newtcolorbox{intuitionbox}{colframe=MediumPurple, colback=purple!10!white, title=Intuition}

% Geometry settings
\geometry{letterpaper, portrait, includeheadfoot=true, hmargin=1in, vmargin=1in}
\onehalfspacing

% Header and footer
\pagestyle{fancy}
\fancyhf{}
\lhead{MAT232 - Lecture Notes}
\rhead{\thepage}
\lfoot{University of Toronto Mississauga}
\rfoot{\today}

% Document starts
\begin{document}
\renewcommand{\familydefault}{\rmdefault}

\begin{titlepage}
    \null % This is a TeX command that does nothing but is necessary for vfill to work correctly
    \vfill
    \begin{center}
        {\fontsize{40}{48}\selectfont \bfseries MAT232 - Lecture 3}
        \vspace{20pt} \\
        {\LARGE Polar Coordinates and the Arc Length of Parametric Curves} \\
        \vspace{20pt}
        \textbf{AlexanderTheMango}
        \vspace{8pt}
        \\ Prepared for January 13, 2025
    \end{center}
    \vfill
\end{titlepage}

\normalsize

\input{preliminary}
\section*{Parametric Equations and Parameters}

\begin{definitionbox}
If \( x \) and \( y \) are continuous functions of \( t \) on an interval \( I \), then the equations  
\[
x = x(t) \quad \text{and} \quad y = y(t)
\]  
are called \textbf{parametric equations}, and \( t \) is called the \textbf{parameter}. The set of points \( (x, y) \) obtained as \( t \) varies over the interval \( I \) is called the \textbf{graph of the parametric equations}. The graph of parametric equations is referred to as a \textbf{parametric curve} or \textbf{plane curve}, and is denoted by \( C \).  
\end{definitionbox}

\section*{Theorem 1.1: Derivative of Parametric Equations}
\begin{theorembox}
Consider the plane curve defined by the parametric equations \( x = x(t) \) and \( y = y(t) \). Suppose that \( x'(t) \) and \( y'(t) \) exist, and assume that \( x'(t) \neq 0 \). Then the derivative \( \frac{dy}{dx} \) is given by  
\[
\frac{dy}{dx} = \frac{\frac{dy}{dt}}{\frac{dx}{dt}} = \frac{y'(t)}{x'(t)}.
\]
\begin{proofbox}
    \begin{proof}
    \leavevmode\\
        This theorem can be proven using the Chain Rule. Assume that the parameter \( t \) can be eliminated, yielding a differentiable function \( y = F(x) \). Then \( y(t) = F(x(t)) \). Differentiating both sides of this equation using the Chain Rule gives  
        \[
        y'(t) = F'(x(t)) x'(t),
        \]  
        so  
        \[
        F'(x(t)) = \frac{y'(t)}{x'(t)}.
        \]  
        But \( F'(x(t)) = \frac{dy}{dx} \), which proves the theorem. \\
    \end{proof}
    \end{proofbox}
\end{theorembox}

\section*{Equation 1.1 and Applications}
\begin{notebox}
Equation 1.1 can be used to calculate derivatives of plane curves, as well as critical points. Recall that a critical point of a differentiable function \( y = f(x) \) is any point \( x = x_0 \) such that either \( f'(x_0) = 0 \) or \( f'(x_0) \) does not exist. Equation 1.1 gives a formula for the slope of a tangent line to a curve defined parametrically regardless of whether the curve can be described by a function \( y = f(x) \) or not.
\end{notebox}

\section*{Second-Order Derivatives}
\begin{theorembox}
The next goal is to see how to take the second derivative of a function defined parametrically. The second derivative of a function \( y = f(x) \) is defined to be the derivative of the first derivative; that is,  
\[
\frac{d^2y}{dx^2} = \frac{d}{dx} \left[ \frac{dy}{dx} \right].
\]  

Since \( \frac{dy}{dx} = \frac{\frac{dy}{dt}}{\frac{dx}{dt}} \), it is possible to replace \( y \) on both sides of this equation with \( \frac{dy}{dx} \). This yields
\[
\frac{d^2y}{dx^2} = \frac{\frac{d}{dt}\left( \frac{dy}{dx} \right)}{\frac{dx}{dt}}.
\]
\end{theorembox}

\input{intolecturecontent}

\section*{Key Concepts}
\begin{definitionbox}
A \textbf{parametric equation} is a set of equations that express the coordinates of the points of a curve as functions of a variable, called a parameter.
\end{definitionbox}

\section*{Examples}
\begin{examplebox}
\textbf{Example:} Sketch the graph, using a table of values:
\[ x = t + \frac{1}{t}, \quad y = t - \frac{1}{t}, \quad t > 0. \]

\begin{center}
\begin{adjustbox}{max width=\textwidth}
\LARGE
\begin{tabular}{|c|c|c|c|}
\hline
$t$    & $1/t$ & $x$ & $y$ \\ \hline
$0.01$ & $\nicefrac{1}{0.01} = \nicefrac{1}{\frac{1}{100}} = 100$     & $100.01$   & $0.01 - 100 = -99.99$   \\ \hline
$0.1$  & $\nicefrac{1}{0.1} = \nicefrac{1}{\frac{1}{10}} = 10$     & $10.1$   & $-9.9$   \\ \hline
$0.2$  & $\nicefrac{1}{0.2} = \nicefrac{1}{\frac{20}{100}} = \nicefrac{1}{\frac{2}{10}} = 5$     & $5.2$   & $4.8$   \\ \hline
$1$    & $\frac{1}{1}$     & $2$   & $0$   \\ \hline
$5.0$  & $0.2$     & $5.2$   & $4.8$   \\ \hline
$10$   & $0.1$     & $10.1$   & $9.9$   \\ \hline
$10$   & $0.01$    & $100.01$   & $99.99$   \\ \hline
\end{tabular}
\normalsize
\end{adjustbox}
\end{center}

This describes a hyperbolic curve.
\end{examplebox}

\begin{figure}[H]
    \centering
    \includegraphics[width=0.6\textwidth]{sample_image.jpg}
    \caption{Sample image illustrating the concept.}
    \label{fig:sample_image}
\end{figure}

\begin{examplebox}
\textbf{Example:} Sketch the graph (this is the same one), using the elimination method:
\[ x = t + \frac{1}{t}, \quad y = t - \frac{1}{t}, \quad t > 0. \]
\( LHS = A^2 - B^2 = (A - B)(A + B) = RHS \)
\( X = A \) and \( y = B \).
\( LHS: x^2 - y^2 \).
\( A - B = x - y = (t + \frac{1}{t}) - (t - \frac{1}{t}) = \frac{2}{t} \).
\( A + B = x + y = (t + \frac{1}{t}) + (t - \frac{1}{t}) = 2t \).
\( RHS: (A - B)(A + B) = (x - y)(x + y) = (\frac{2}{t})(2t) = 4 \).
Therefore, \( x^2 - y^2 = 4 \), \( y \in \mathbb{R} \) will work, \( x > 0 \).

This describes a hyperbolic curve.
\end{examplebox}

\section*{Theorems and Proofs}
\begin{theorembox}
\textbf{Theorem:} If $x(t)$ and $y(t)$ are differentiable functions, the slope of the curve is given by:
\[ \frac{dy}{dx} = \frac{\frac{dy}{dt}}{\frac{dx}{dt}}, \quad \text{provided } \frac{dx}{dt} \neq 0. \]
\end{theorembox}

\begin{figure}[H]
    \centering
    \includegraphics[width=0.6\textwidth]{sample_image1.jpg}
    \caption{Graphical representation of the theorem.}
    \label{fig:sample_image1}
\end{figure}

\section*{Practice Questions}
\begin{notebox}
Try this question at home! \\
\\
Sketch and eliminate \( t \) if possible: \\
\[
    x = t^2, \quad y = t^3, \quad -2 \leq t \leq 2
\]
Note that this is a closed interval. The starting point is the smallest value of \( t \). This highlights where the graph should begin. The finishing point should be the largest value of \( t \). \\
Using an arrow, make sure to indicate the direction of the graph as \( t \to \infty \). \\
\end{notebox}

\begin{notebox}
Try another question at home! \\
\\
Sketch and eliminate \( t \) if possible: \\
\[
    c_1: x = -cos(\frac{t}{4}), y = sin(\frac{t}{4}), for 0 \leq t \leq 4\pi
\]
\[
    c_2: x = -sin(t), y = -cos(t), for \frac{\pi}{2} \leq t \leq \frac{3\pi}{2}
\]
\[
    c_3: x = cos(t), y = sin(t), for t \in [0, \pi]
\]
Hint: \( x = rcos(\theta), y = rsin(\theta), x^2 + y^2 = r^2 \). Also, for these curves, it follows that \( r = 1 \).
\end{notebox}

\section*{The Elimination Method Does NOT Always Work}
\begin{notebox}
Consider the following case where \( t \) cannot be eliminated:
\[
    x = e^t - \sin^2(t), \quad y = ln(t) + \frac{1}{t}, \quad t > 0
\]
\end{notebox}

\section*{Further Visualization}
\begin{figure}[H]
    \centering
    \includegraphics[width=0.6\textwidth]{sample_image2.jpg}
    \caption{Additional visualization for parametric curves.}
    \label{fig:sample_image2}
\end{figure}

\section*{Section 1.2: Calculus on Parametric Equations}
\section*{Key Concepts}
Recall the concept from $1^{\text{st}}$ year calculus:
\begin{definitionbox}
If \( y = f(x) \) is given, then the slope of the tangent line to the curve of \( y = f(x) \) is:
\[
    y' = f'(x) = \frac{dy}{dx}
\]
\end{definitionbox}
Now, for MAT232, we have:
\begin{definitionbox}
Given \( x = f(t), \quad y = g(t), \quad t \in \mathbb{R} \), these are defifferentiable w.r.t.\ (w.r.t.\ = ``with respect to'') \( t \). This is such that:
\[
    \frac{dy}{dx} = \frac{\frac{dy}{dt}}{\frac{dx}{dt}}, \frac{dx}{dt} \neq 0
\]
This will also be provided in the formula sheet. \\
\\
\[
    x = f(t), \quad y = g(t), \quad t \in \mathbb{R}
\]
Because the chain rule must follow through, always! \\
\textit{Here is the derivation:}
So \dots \( y = g(t) \) \\
\[
     \frac{dy}{dt} = \frac{dy}{dx} \cdot \frac{dx}{dt}
\]
\textit{Chain rule.} \\
\\
\[
    \frac{\frac{dy}{dt}}{\frac{dx}{dt}} = \frac{dy}{dx}, \quad \text{provided that} \quad \frac{dx}{dt} \neq 0
\]
\end{definitionbox}

\section*{Second Derivative}
\begin{theorembox}
Given \( x = f(t), y = g(t), t \in \mathbb{R} \) are differentiable at \( t \) and \( \frac{dy}{dx} = \frac{\frac{dy}{dt}}{\frac{dx}{dt}} \) exists and is differentiable at \( t \):
\[
    \frac{d^{2y}}{dx^2} = \frac{d}{dx}(\frac{dy}{dx}) = dx(\frac{\frac{dy}{dt}}{\frac{dx}{dt}})
\]
Notice that the expression of the innermost bracket is a derivative all in terms of \( t \).
Thus:
\[
    = \frac{d}{dt}(\frac{\frac{dy}{dt}}{\frac{dx}{dt}}) \cdot \frac{dt}{dx} = \frac{d}{dt}(\frac{\frac{dy}{dt}}{\frac{dx}{dt}}) = \frac{\frac{d}{dt}(\frac{\frac{dy}{dt}}{\frac{dx}{dt}})}{\frac{dx}{dt}} \text{.}
\]
This follows from the \textbf{inverse function theorem}. \\
Collectively, it follows that:
\[
    \frac{d^2y}{dx^2} = \frac{\frac{d}{dt}(\frac{\frac{dy}{dt}}{\frac{dx}{dt}})}{\frac{dx}{dt}}, \quad \frac{dx}{dt} \neq 0 \text{.}
\]
\textit{This is not included on the formula sheet.}
\end{theorembox}

\section*{Examples}
\begin{examplebox}
Consider the following parametric curve:
\[
    x = \sec(t), \quad y = \tan(t), \quad -\frac{\pi}{2} < t < \frac{\pi}{2}
\]
(A) Find the tangent line to the given curve at the point \( (\sqrt{2}, 1) \) where \( t = \frac{\pi}{4} \). \\
(B) Find the vertical tangent(s), if any. \\
(C) Find \( \frac{d^2y}{dx^2} \).
\end{examplebox}
Let's do this, one at a time! \\
\\
(A) Find the tangent line to the given curve at the point \( (\sqrt{2}, 1) \) where \( t = \frac{\pi}{4} \).
\begin{examplebox}
\underline{Tangent Line:}
Recall\dots
\begin{enumerate}
    \item \( y - y_0 = m(x - x_0) \), where \( m \) is the slope and \( (x_0, y_0) \) is a point on the curve;
    \item \( y = mx + b \), where \( m \) is the slope and \( b \) is the y-intercept.
\end{enumerate}
Given point \( (\sqrt{2}, 1) = (x_0, y_0) \), \( \frac{dy}{dt} = \sec^2(t) \), and \( \frac{dx}{dt} = \sec(t)\tan(t) \), it follows that:
\[
    m = \frac{dy}{dx} = \frac{\frac{dy}{dt}}{\frac{dx}{dt}} = \frac{\sec^{\cancel{2}}(t)\tan(t)}{\cancel{\sec(t)}\tan(t)} = \frac{\sec(t)}{\tan(t)}
\]
Next, \( \frac{dy}{dx} \mid_{t = \frac{\pi}{4}} = \frac{\sec(\frac{\pi}{4})}{\tan(\frac{\pi}{4})} = \frac{\sqrt{2}}{1} = \sqrt{2} = m \). \\

\textbf{self-note: finish these notes (check the camera roll)}

\end{examplebox}

(B) Find the vertical tangent(s), if any.
\begin{examplebox}
\[
    \frac{dy}{dt} = \sec^2(t)
\]
\[
    \frac{dx}{dt} = \sec(t)\tan(t)
\]
So\dots
\[
    \frac{dy}{dx} = \frac{\frac{dy}{dt}}{\frac{dx}{dt}} = \frac{\sec^2(t)}{\sec(t)\tan(t)}
\]

Recall from first year calculus:
\begin{theorembox}
    Given \( y = f(x) \), it follows that \( y' = f'(x) = 0 \). That is, the roots of \( y' = 0 \) indicate the positions of the horizontal tangents.
\end{theorembox}
So\dots\\
Horizonal Tangent: \( \frac{dy}{dx} = 0 \); find \( t \) values.
\[
    \frac{dy}{dt} = 0, \quad \text{but } \frac{dx}{dt} \neq 0
\]
Vertical Tangent: \( \frac{dy}{dx} \) is \textit{undefined}; find \( t \) values.
\[
    \frac{dx}{dt} = 0, \quad \text{but} \frac{dy}{dt} \neq 0
\]
In this case, there is a singular point:
\[
    \frac{dx}{dt} = 0 \quad \text{and} \quad \frac{dy}{dt} = 0
\]
Vertical Tangents: \( \frac{dx}{dt} = 0 \), but \( \frac{dy}{dt} \neq 0 \). \\
So\dots
\[
    \frac{dx}{dt} = \sec(t)\tan(t) = 0, \quad -\frac{\pi}{2} < t < \frac{\pi}{2} \text{.}
\]
Notice that
\begin{itemize}
    \item \( \sec(t) = \frac{1}{\cos(t)} = 0 \) is impossible as \( 1 \neq 0 \);
    \item \( \tan(t) = 0 \) occurs at \( t = 0 \).
\end{itemize}
Now, check \( \frac{dy}{dt} = 0 \) at \( t = 0 \).
\[
    \frac{dy}{dt} = \sec^2(t) = 0, \quad \text{for } t = 0
\]
Is this true? \\
\\
Therefore, the vertical tangent is at \( t = 0 \).
\end{examplebox}

(C) Find \( \frac{d^2y}{dx^2} \).
\begin{examplebox}
Recall:
\[
    \frac{d^2y}{dx^2} = \frac{\frac{d}{dt}(\frac{\frac{dy}{dt}}{\frac{dx}{dt}})}{\frac{dx}{dt}}
\]
\[
    \frac{dy}{dx} = \frac{\sec(t)}{\tan(t)} \quad \text{and} \quad \frac{dx}{dt} = \sec(t)\tan(t)
\]
\[
    \frac{d^2y}{dx^2} = \frac{\frac{d}{dt}(\frac{\sec(t)}{\tan(t)})}{\sec(t)\tan(t)}
\]
\[
    \frac{\sec(t)}{\tan(t)} = \frac{\frac{1}{cos(t)}} \cdot {\frac{\sin(t)}{\cos(t)}} = \frac{1}{\cos(t)}(\frac{\cos(t)}{\sin(t)})
\]
\[
    = \frac{1}{\sin(t)}
\]
\[
    \sec(t)\tan(t) = \frac{1}{\cos(t)} \cdot \frac{\sin(t)}{\cos(t)} = \frac{\sin(t)}{\cos^2(t)}
\]
Now, find the derivative of \( y = \frac{1}{\sin(t)}\):
\[
    y' = \frac{0 \cdot \sin(t) - \cos(t) \cdot 1}{\sin^2(t)} = -\frac{\cos(t)}{\sin^2(t)}
\]
\textbf{note to self: finish this off}
\end{examplebox}

\end{document}
