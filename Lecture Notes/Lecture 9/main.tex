\documentclass{article}
\usepackage{amsthm,amsmath,amsfonts,lipsum}
\usepackage[T1]{fontenc}
\usepackage{beramono}
\usepackage{listings}
\usepackage{fontawesome5}
\usepackage{adjustbox}
\usepackage{mathabx}
\usepackage{thmtools}
\usepackage{import}
\usepackage{graphicx}
\usepackage{setspace}
\usepackage{geometry}
\usepackage{physics}
\usepackage{float}
\usepackage[english]{babel}
\usepackage{framed}
\usepackage[dvipsnames,x11names]{xcolor}
\usepackage{tcolorbox}
\usepackage{fancyhdr}
\usepackage{hyperref}
\usepackage{booktabs}
\usepackage{enumitem}
\usepackage{cancel}
\usepackage{background}
\usepackage{units}

% Configuring the background
\backgroundsetup{
  scale=1, % Optional, scale if needed
  color=black, % Optional, set the image color, can be omitted
  opacity=0.18, % Optional, adjust opacity for watermark effect
  angle=0,
  position=current page.center, % Center the image on the page
  contents={\includegraphics[width=1.75\paperwidth, height=1.75\paperheight, keepaspectratio]{ninym_ralei_leaf (watermarked by AlexanderTheMango)}} % Keeps aspect ratio and scales to fill the page
}

% Colours
\definecolor{darkgreen}{rgb}{0.0, 0.5, 0.0}
\definecolor{Firebrick}{rgb}{0.698, 0.132, 0.203}
\definecolor{Crimson}{rgb}{0.862745, 0.078431, 0.235294} % Crimson color
\definecolor{lightred}{rgb}{1.0, 0.819608, 0.819608} % Light red for background
\definecolor{MediumPurple}{rgb}{0.576, 0.439, 0.859}
\definecolor{chocolate}{rgb}{0.82, 0.41, 0.12} % Chocolate color definition
% Define the Navy color
\definecolor{Navy}{rgb}{0.0, 0.0, 0.5}

% Define custom tcolorbox styles for notes
\tcbuselibrary{skins, breakable}
\newtcolorbox{definitionbox}{colframe=RoyalBlue, colback=blue!5!white, title=Definition}
\newtcolorbox{examplebox}{colframe=ForestGreen, colback=green!5!white, title=Example}
\newtcolorbox{notebox}{colframe=RedOrange, colback=orange!5!white, title=Note}
\newtcolorbox{theorembox}{colframe=RoyalPurple, colback=purple!5!white, title=Theorem}

\newtcolorbox{propositionbox}{colframe=Goldenrod, colback=yellow!10!white, title=Proposition}
\newtcolorbox{remarkbox}{colframe=MidnightBlue, colback=blue!10!white, title=Remark}
\newtcolorbox{corollarybox}{colframe=OliveGreen, colback=green!10!white, title=Corollary}
\newtcolorbox{warningbox}{colframe=Crimson, colback=lightred, title=Warning}
\newtcolorbox{proofbox}{colframe=Black, colback=gray!10!white, title=Proof}
\newtcolorbox{questionbox}{colframe=Teal, colback=teal!10!white, title=Question}
\newtcolorbox{tipbox}{colframe=Goldenrod, colback=yellow!10!white, title=Tip}
\newtcolorbox{exercisebox}{colframe=darkgreen, colback=green!5!white, title=Exercise}
\newtcolorbox{solutionbox}{colframe=DodgerBlue4, colback=blue!5!white, title=Solution}
\newtcolorbox{algorithmbox}{colframe=Navy, colback=blue!10!white, title=Algorithm}
\newtcolorbox{conceptbox}{colframe=chocolate, colback=brown!10!white, title=Concept}
\newtcolorbox{illustrationbox}{colframe=Firebrick, colback=red!10!white, title=Illustration}
\newtcolorbox{intuitionbox}{colframe=MediumPurple, colback=purple!10!white, title=Intuition}
\newtcolorbox{answerbox}{colframe=RoyalBlue, colback=blue!10!white, title=Answer}

% Geometry settings
\geometry{letterpaper, portrait, includeheadfoot=true, hmargin=1in, vmargin=1in}
\onehalfspacing

% Header and footer
\pagestyle{fancy}
\fancyhf{}
\lhead{MAT232 - Lecture Notes}
\rhead{\thepage}
\lfoot{University of Toronto Mississauga}
\rfoot{\today}

% Document starts
\begin{document}
\renewcommand{\familydefault}{\rmdefault}

\begin{titlepage}
    \null % This is a TeX command that does nothing but is necessary for vfill to work correctly
    \vfill
    \begin{center}
        {\fontsize{40}{48}\selectfont \bfseries MAT232 - Lecture 3}
        \vspace{20pt} \\
        {\LARGE Polar Coordinates and the Arc Length of Parametric Curves} \\
        \vspace{20pt}
        \textbf{AlexanderTheMango}
        \vspace{8pt}
        \\ Prepared for January 13, 2025
    \end{center}
    \vfill
\end{titlepage}


\setcounter{page}{0}
\newpage
\tableofcontents
\newpage

\input{preliminary}



\input{intolecturecontent}
\normalsize

\setcounter{page}{1}

\begin{examplebox}
If \( \overline{r}''(t) = \langle 2t, 3 \rangle \), find \( \overline{r}'(t) \) with \( \overline{r}'(0) = \langle 1, 0 \rangle \) and \( \overline{r}(0) = \langle -1, 1 \rangle \).
\begin{solutionbox}
We know that \( \overline{r}''(t) = \overline{r}'(t) \). Therefore, we have
\begin{align*}
    \overline{r}'(t) &= \int \overline{r}''(t) \dd{t} \\
    &= \int \langle 2t, 3 \rangle \dd{t} \\
    &= \langle t^2 + C_1, 3t + C_2 \rangle.
\end{align*}
We can now use the initial conditions to solve for \( C_1 \) and \( C_2 \). We have
\begin{align*}
    \overline{r}'(0) &= \langle 0 + C_1, 0 + C_2 \rangle = \langle 1, 0 \rangle \\
    \overline{r}(0) &= \langle -1, 1 \rangle.
\end{align*}
Therefore, we have
\begin{align*}
    C_1 &= 1 \\
    C_2 &= 0.
\end{align*}
Therefore, the solution is
\begin{equation*}
    \overline{r}'(t) = \langle t^2 + 1, 3t \rangle.
\end{equation*}
\begin{answerbox}
The solution is \( \overline{r}'(t) = \langle t^2 + 1, 3t \rangle \).
\end{answerbox}
\end{solutionbox}
\end{examplebox}
So,
\[
    \int \overline{r}'(t) dt = \int \langle t^2 + 1, 3t \rangle dt = \langle \frac{t^3}{3} + t + C_3, \frac{3t^2}{2} + C_4 \rangle.
\]
We have that
\[
    \overline{r}(t) = \langle \frac{t^3}{3} + t + C_3, \frac{3t^2}{2} + C_4 \rangle.
\]
We can now use the initial conditions to solve for \( C_3 \) and \( C_4 \). We have
\begin{align*}
    \overline{r}(0) &= \langle \frac{0}{3} + 0 + C_3, \frac{3 \cdot 0^2}{2} + C_4 \rangle = \langle -1, 1 \rangle.
\end{align*}
Therefore, we have
\begin{align*}
    C_3 &= -1 \\
    C_4 &= 1.
\end{align*}
Therefore, the solution is
\[
    \overline{r}(t) = \langle \frac{t^3}{3} + t - 1, \frac{3t^2}{2} + 1 \rangle.
\]

\subsection*{Find the Length of a Vector-Valued Function}
\addcontentsline{toc}{subsection}{Find the Length of a Vector-Valued Function}

\begin{definitionbox}
Parametric Equation:
\[
    x = f(t), \quad y = g(t), \quad \alpha \leq t \leq 
\]
\[
    L = \int_{\alpha}^{\beta} \sqrt{[f'(t)]^2 + [g'(t)]^2} \dd{t}
\]
\[
    L = \int_{\alpha}^{\beta} \sqrt{\left[\frac{dx}{dt}\right]^2 + \left[\frac{dy}{dt}\right]^2} \dd{t}
\]
\[
    \text{Length of the curve} = \int_{\alpha}^{\beta} \sqrt{[x'(t)]^2 + [y'(t)]^2} \dd{t}
\]
\end{definitionbox}

\subsection*{Now, for Vector-Valued Functions}
\addcontentsline{toc}{subsection}{Now, for Vector-Valued Functions}
\begin{definitionbox}
Vector-Valued Functions:
\begin{align*}
    \overline{r}(t) = \langle x(t), y(t), z(t) \rangle, \quad \alpha \leq t \leq \beta \\
    &= \langle f(t), g(t), h(t) \rangle \\
    &= f(t) \hat{i} + g(t) \hat{j} + h(t) \hat{k} \\
\end{align*}
\begin{align*}
    \overline{r}'(t) = \langle f'(t), g'(t), h'(t) \rangle
    &= \frac{d\overline{r}}{dt} = \frac{d}{dt} \langle f(t), g(t), h(t) \rangle \\
    &= \frac{df}{dt} \hat{i} + \frac{dg}{dt} \hat{j} + \frac{dh}{dt} \hat{k}
\end{align*}
Thus,
\begin{align*}
    \mid \mid \overline{r}'(t) \mid \mid &= \sqrt{[f'(t)]^2 + [g'(t)]^2 + [h'(t)]^2} \\
    &= \sqrt{\left[\frac{dx}{dt}\right]^2 + \left[\frac{dy}{dt}\right]^2 + \left[\frac{dz}{dt}\right]^2} \\
    &= \sqrt{[x'(t)]^2 + [y'(t)]^2 + [z'(t)]^2}.
\end{align*}
So, the length of the curve is
\[
    L = \int_{\alpha}^{\beta} \sqrt{\left[\frac{dx}{dt}\right]^2 + \left[\frac{dy}{dt}\right]^2 + \left[\frac{dz}{dt}\right]^2} \dd{t}.
\]
\end{definitionbox}

\begin{examplebox}
Find the length of the curve defined by the vector-valued function
\[
    \overline{r}(t) = \langle cos(t), sin(t), t \rangle, \quad 0 \leq t \leq \pi.
\]
\[
    x = cos(t), \quad y = sin(t), \quad z = t, \quad 0 \leq t \leq \pi
\]
\begin{solutionbox}
We have
\begin{align*}
    \overline{r}'(t) &= \langle -sin(t), cos(t), 1 \rangle \\
    \mid \mid \overline{r}'(t) \mid \mid &= \sqrt{(-sin(t))^2 + (cos(t))^2 + 1^2} \\
    &= \sqrt{sin^2(t) + cos^2(t) + 1} \\
    &= \sqrt{1 + 1} \\
    &= \sqrt{2}.
\end{align*}
Therefore, the length of the curve is
\begin{align*}
    L &= \int_{0}^{\pi} \sqrt{2} \dd{t} \\
    &= \sqrt{2} \int_{0}^{\pi} \dd{t} \\
    &= \sqrt{2} \left[ t \right]_{0}^{\pi} \\
    &= \sqrt{2} \pi.
\end{align*}
\begin{answerbox}
The length of the curve is \( \sqrt{2} \pi \).
\end{answerbox}
\end{solutionbox}
\end{examplebox}

\section*{Section 4.1: Functions of Several Variables}
\addcontentsline{toc}{section}{Section 4.1: Functions of Several Variables}
\begin{conceptbox}
Function of one variable:
\[
    y = f(x), \quad \text{where } x \text{ is the independent variable and } y \text{ is the dependent variable}.
\]
The image/range of the function is the set of all possible values of \( y \) as \( x \) varies over the domain \( \mathbb{R} \) of the function.
\begin{notebox}
These functions are in 2D space.
\begin{itemize}
    \item \( f: x \in A \to y \in B \) is a function from set \( A \) to set \( B \).
    \item The domain of the function is the set of all possible values of \( x \) for which the function is defined.
    \item The range of the function is the set of all possible values of \( y \) as \( x \) varies over the domain.
    \item The graph of the function is the set of all points \( (x, y) \) in the plane such that \( y = f(x) \).
    \item The level curves of the function are the curves in the plane defined by \( f(x, y) = k \), where \( k \) is a constant.
    \item The level surfaces of the function are the surfaces in space defined by \( f(x, y, z) = k \), where \( k \) is a constant.
    \item \( \mathbb{R} \to \mathbb{R} \).
\end{itemize} 
\end{notebox}
\end{conceptbox}

Now, we will extend this concept to functions of several variables.
\begin{definitionbox}
Function of two variables:
\[
    z = f(x, y), \quad \text{where } x \text{ and } y \text{ are the independent variables and } z \text{ is the dependent variable}.
\]
The domain of the function is the set of all possible values of \( (x, y) \) for which the function is defined.
\begin{notebox}
These functions are in 3D space.
\begin{itemize}
    \item \( f: (x, y) \in A \to z \in B \) is a function from set \( A \) to set \( B \).
    \item The domain of the function is the set of all possible values of \( (x, y) \) for which the function is defined.
    \item The range of the function is the set of all possible values of \( z \) as \( (x, y) \) varies over the domain.
    \item The graph of the function is the set of all points \( (x, y, z) \) in space such that \( z = f(x, y) \).
    \item The level curves of the function are the curves in the plane defined by \( f(x, y) = k \), where \( k \) is a constant.
    \item The level surfaces of the function are the surfaces in space defined by \( f(x, y, z) = k \), where \( k \) is a constant.
    \item \( \mathbb{R}^2 \to \mathbb{R} \).
\end{itemize}
\end{notebox}
\end{definitionbox}

There are even functions of three variables.
\begin{definitionbox}
Function of three variables:
\[
    w = f(x, y, z), \quad \text{where } x, y, \text{ and } z \text{ are the independent variables and } w \text{ is the dependent variable}.
\]
The domain of the function is the set of all possible values of \( (x, y, z) \) for which the function is defined.
\begin{notebox}
These functions are in 4D space.
\begin{itemize}
    \item \( f: (x, y, z) \in A \to w \in B \) is a function from set \( A \) to set \( B \).
    \item The domain of the function is the set of all possible values of \( (x, y, z) \) for which the function is defined.
    \item The range of the function is the set of all possible values of \( w \) as \( (x, y, z) \) varies over the domain.
    \item The graph of the function is the set of all points \( (x, y, z, w) \) in space such that \( w = f(x, y, z) \).
    \item The level curves of the function are the curves in space defined by \( f(x, y, z) = k \), where \( k \) is a constant.
    \item \( \mathbb{R}^3 \to \mathbb{R} \).
    \item The level surfaces of the function are the surfaces in space defined by \( f(x, y, z) = k \), where \( k \) is a constant.
    \item \( \mathbb{R}^3 \to \mathbb{R} \).
    \item The graph of the function is the set of all points \( (x, y, z, w) \) in space such that \( w = f(x, y, z) \).
    \item The level curves of the function are the curves in space defined by \( f(x, y, z) = k \), where \( k \) is a constant.
    \item The level surfaces of the function are the surfaces in space defined by \( f(x, y, z) = k \), where \( k \) is a constant.
    \item \( \mathbb{R}^3 \to \mathbb{R} \).
\end{itemize}
\end{notebox}
\end{definitionbox}

\subsection*{Sketching a Function in 2D}
\addcontentsline{toc}{subsection}{Sketching a Function in 2D}
\begin{examplebox}
Let \( f(x, y) = 3x\sqrt{y} - 1 \). Sketch the graph of the function.
\begin{solutionbox}
\begin{notebox}
\begin{itemize}
    \item \( x \in \mathbb{R} \)
    \item \( \sqrt{y} \) so \( y \geq 0 \) 
\end{itemize}
\end{notebox}
Consider some points in the domain of the function. We have
\begin{align*}
    x, y &= f(x, y) \\
    f(1, 4) &= 3 \cdot 1 \cdot 2 - 1 = 5 = z \\
    f(0, 9) &= 3 \cdot 0 \cdot 3 - 1 = -1 = z \\
    f(t, t^2) &= 3 \cdot t \cdot \sqrt{t^2} - 1 = 3t \cdot t - 1 = 3t^2 - 1 = z.
\end{align*}
\begin{notebox}
\( \sqrt{t^2} = \mid t \mid \).
\end{notebox}

Therefore, the function is defined for all \( (x, y) \) in the domain.
\\
To sketch the graph of the function, we can sketch the level curves of the function. We have
\[
    f(x, y) = 3x\sqrt{y} - 1 = k.
\]
We can now sketch the level curves of the function by setting \( k = 0 \). We have
\[
    3x\sqrt{y} - 1 = 0.
\]
Therefore, the level curve is
\[
    3x\sqrt{y} = 1.
\]
\begin{center}
    \includegraphics[width=0.5\textwidth]{sample_image.jpg}
\end{center}
\begin{answerbox}
The graph of the function is shown above.
\end{answerbox}
\end{solutionbox}
\end{examplebox}

\subsection*{Sketching the Domain of a Function in 2D}
\addcontentsline{toc}{subsection}{Sketching the Domain of a Function in 2D}
\begin{examplebox}
    Sketch the domain of \( f(x, y) = \ln(x^2 - y) \).
\begin{solutionbox}
\begin{notebox}
\begin{itemize}
    \item \( \ln(x^2 - y) \) implies that \( x^2 - y > 0\) meaning that \( x^2 > y \).
    \item \( x^2 - y > 0 \) so \( x^2 > y \).
    \item \( x^2 - y \) is in the domain of \( \ln(x^2 - y) \) so \( x^2 - y > 0 \).
    \item \( x^2 - y > 0 \) so \( y < x^2 \).
    \item The domain of the function is \( y < x^2 \).
\end{itemize}
\end{notebox}
Therefore, the domain of the function is \( y < x^2 \).
Pick some points in the domain of the function. We have
\begin{align*}
    f(2, 0) &= \ln(2^2 - 0) = \ln(4) \\
    f(0, 2) &= \ln(0^2 - 2) = \ln(-2).
\end{align*}
Notice that \( f(0, 2) \) is not defined because \( \ln(-2) \) is not defined.
\begin{illustrationbox}
\begin{center}
    \includegraphics[width=0.5\textwidth]{sample_image.jpg}
\end{center}
\end{illustrationbox}
\begin{answerbox}
The domain of the function is \( y < x^2 \).
\end{answerbox}
\end{solutionbox}
\end{examplebox}

\subsection*{Describing the Domain of a Function in 3D}
\addcontentsline{toc}{subsection}{Describing the Domain of a Function in 3D}
\begin{examplebox}
Describe the domain of the function \( f(x, y, z) = \sqrt{1-x^2-y^2-z^2} \).
\begin{solutionbox}
\begin{notebox}
\begin{itemize}
    \item \( \sqrt{1-x^2-y^2-z^2} \) is defined if \( 1-x^2-y^2-z^2 \geq 0 \).
    \item \( 1-x^2-y^2-z^2 \geq 0 \) so \( 1 \geq x^2 + y^2 + z^2 \).
    \item The domain of the function is \( x^2 + y^2 + z^2 \leq 1 \).
    \item The domain of the function is the unit ball.
    \item The domain of the function is the set of all points \( (x, y, z) \) such that \( x^2 + y^2 + z^2 \leq 1 \).
    \item The domain of the function is the unit ball.
\end{itemize}
\end{notebox}
Notice that
\begin{align*}
\sqrt{1-x^2-y^2-z^2} &> 0 \\
1-x^2-y^2-z^2 &> 0 \\
1 &> x^2 + y^2 + z^2.
\end{align*}
This means, any point \( (x, y, z) \) on the edge of the unit ball/sphere centered at the origin (0, 0, 0) with radius 1 and inside the unit ball is in the domain of the function.
\begin{align*}
    f(0, \frac{1}{2}, -\frac{1}{2}) &= \sqrt{1 - 0 - \left(\frac{1}{2}\right)^2 - \left(-\frac{1}{2}\right)^2} \\
    &= \sqrt{1 - \frac{1}{4} - \frac{1}{4}} \\
    &= \sqrt{1 - \frac{1}{2}} \\
    &= \sqrt{\frac{1}{2}} \\
    &= \frac{1}{\sqrt{2}} \\
    &= \frac{\sqrt{2}}{2}.
\end{align*}
Therefore, the domain of the function is \( x^2 + y^2 + z^2 \leq 1 \).
\begin{answerbox}
The domain of the function is \( x^2 + y^2 + z^2 \leq 1 \).
\end{answerbox}
\end{solutionbox}
\end{examplebox}

\subsection*{Graphing and Describing Functions}
\addcontentsline{toc}{subsection}{Graphing and Describing Functions}
\begin{examplebox}
Graph the following functions and describe them:
\begin{enumerate}
    \item \( f(x, y) = 1 - x - \frac{y}{2} \)
    \item \( f(x, y) = \sqrt{1 - x^2 - y^2} \)
    \item \( f(x, y) = -\sqrt{x^2 + y^2} \)
\end{enumerate}
\begin{solutionbox}
\begin{enumerate}
    \item The function is a plane. The graph of the function is a plane with a slope of -1 in the x-direction and -1/2 in the y-direction.
    \item The function is a cone. The graph of the function is a cone with a vertex at the origin and a base of radius 1 in the xy-plane.
    \item The function is a cone. The graph of the function is a cone with a vertex at the origin and a base of radius 1 in the xy-plane.
\end{enumerate}
\begin{answerbox}
\begin{enumerate}
    \item The graph of the function is a plane with a slope of -1 in the x-direction and -1/2 in the y-direction.
    \item The graph of the function is a cone with a vertex at the origin and a base of radius 1 in the xy-plane.
    \item The graph of the function is a cone with a vertex at the origin and a base of radius 1 in the xy-plane.
    \end{enumerate}
\end{answerbox}
\end{solutionbox}
\end{examplebox}

\end{document}
