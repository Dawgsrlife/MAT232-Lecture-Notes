\documentclass{article}
\usepackage{amsthm,amsmath,amsfonts,lipsum}
\usepackage[T1]{fontenc}
\usepackage{beramono}
\usepackage{listings}
\usepackage{fontawesome5}
\usepackage{adjustbox}
\usepackage{mathabx}
\usepackage{thmtools}
\usepackage{import}
\usepackage{graphicx}
\usepackage{setspace}
\usepackage{geometry}
\usepackage{physics}
\usepackage{float}
\usepackage[english]{babel}
\usepackage{framed}
\usepackage[dvipsnames,x11names]{xcolor}
\usepackage{tcolorbox}
\usepackage{fancyhdr}
\usepackage{hyperref}
\usepackage{booktabs}
\usepackage{enumitem}
\usepackage{cancel}
\usepackage{background}
\usepackage{units}
\usepackage{textcomp}

% Configuring the background
\backgroundsetup{
  scale=1, % Optional, scale if needed
  color=black, % Optional, set the image color, can be omitted
  opacity=0.18, % Optional, adjust opacity for watermark effect
  angle=0,
  position=current page.center, % Center the image on the page
  contents={\includegraphics[width=1.75\paperwidth, height=1.75\paperheight, keepaspectratio]{ninym_ralei_leaf (watermarked by AlexanderTheMango)}} % Keeps aspect ratio and scales to fill the page
}

% Colours
\definecolor{darkgreen}{rgb}{0.0, 0.5, 0.0}
\definecolor{Firebrick}{rgb}{0.698, 0.132, 0.203}
\definecolor{Crimson}{rgb}{0.862745, 0.078431, 0.235294} % Crimson color
\definecolor{lightred}{rgb}{1.0, 0.819608, 0.819608} % Light red for background
\definecolor{MediumPurple}{rgb}{0.576, 0.439, 0.859}
\definecolor{chocolate}{rgb}{0.82, 0.41, 0.12} % Chocolate color definition

% Define custom tcolorbox styles for notes
\tcbuselibrary{skins, breakable}
\newtcolorbox{definitionbox}{colframe=RoyalBlue, colback=blue!5!white, title=Definition}
\newtcolorbox{examplebox}{colframe=ForestGreen, colback=green!5!white, title=Example}
\newtcolorbox{notebox}{colframe=RedOrange, colback=orange!5!white, title=Note}
\newtcolorbox{theorembox}{colframe=RoyalPurple, colback=purple!5!white, title=Theorem}

\newtcolorbox{propositionbox}{colframe=Goldenrod, colback=yellow!10!white, title=Proposition}
\newtcolorbox{remarkbox}{colframe=MidnightBlue, colback=blue!10!white, title=Remark}
\newtcolorbox{corollarybox}{colframe=OliveGreen, colback=green!10!white, title=Corollary}
\newtcolorbox{warningbox}{colframe=Crimson, colback=lightred, title=Warning}
\newtcolorbox{proofbox}{colframe=Black, colback=gray!10!white, title=Proof}
\newtcolorbox{questionbox}{colframe=Teal, colback=teal!10!white, title=Question}
\newtcolorbox{tipbox}{colframe=Goldenrod, colback=yellow!10!white, title=Tip}
\newtcolorbox{exercisebox}{colframe=darkgreen, colback=green!5!white, title=Exercise}
\newtcolorbox{solutionbox}{colframe=DodgerBlue4, colback=blue!5!white, title=Solution}
\newtcolorbox{algorithmbox}{colframe=Navy, colback=blue!10!white, title=Algorithm}
\newtcolorbox{conceptbox}{colframe=chocolate, colback=brown!10!white, title=Concept}
\newtcolorbox{illustrationbox}{colframe=Firebrick, colback=red!10!white, title=Illustration}
\newtcolorbox{intuitionbox}{colframe=MediumPurple, colback=purple!10!white, title=Intuition}
\newtcolorbox{answerbox}{colframe=RoyalBlue, colback=blue!10!white, title=Answer}

% Geometry settings
\geometry{letterpaper, portrait, includeheadfoot=true, hmargin=1in, vmargin=1in}
\onehalfspacing

% Header and footer
\pagestyle{fancy}
\fancyhf{}
\lhead{MAT232 - Lecture Notes}
\rhead{\thepage}
\lfoot{University of Toronto Mississauga}
\rfoot{\today}

% Document starts
\begin{document}
\renewcommand{\familydefault}{\rmdefault}

\begin{titlepage}
    \null % This is a TeX command that does nothing but is necessary for vfill to work correctly
    \vfill
    \begin{center}
        {\fontsize{40}{48}\selectfont \bfseries MAT232 - Lecture 3}
        \vspace{20pt} \\
        {\LARGE Polar Coordinates and the Arc Length of Parametric Curves} \\
        \vspace{20pt}
        \textbf{AlexanderTheMango}
        \vspace{8pt}
        \\ Prepared for January 13, 2025
    \end{center}
    \vfill
\end{titlepage}

\input{preliminary}
\input{intolecturecontent}
\normalsize

\section*{Recall the content from last lecture\dots}
\begin{notebox}
Converting from cartesian coordniates \( (x, y) \) to polar coordinates \( r, \theta \).
\[
    x^2 + y^2 = r^2
\]
\[
    \arctan(\dfrac{y}{x}) = \theta
\]
Converting from polar coordinates \( (r, \theta) \) to cartesian coordinates \( (x, y) \) 
\[
    x = r\cos\theta
\]
\[
    y = r\sin\theta
\]
Remember how to between degrees and radians:
\begin{itemize}
    \item Degrees to radians: [fill this in]
    \item Radians to degrees: [fill this in]
\end{itemize}
\end{notebox}

\section*{New}
\begin{conceptbox}
Note the convention for \( r \) in a polar-coordinate point: \\
PC is represented as \( (r, \theta) \).
\[
    (-r, \theta) = (r, \theta + 180\text{\textdegree})
\]
\end{conceptbox}

\subsection*{Example of Plot Points}
\begin{examplebox}
Plot points: \( (3, -45\text{\textdegree}), (3, 225\text{\textdegree}), (4, 330\text{\textdegree}), (1, -45\text{\textdegree}) \)
\textbf{self-note: finish this part up}


\textbf{self-note: add drawing from prof from camera roll here}
\end{examplebox}

\section*{Example of Converting from Polar Coordinates to Cartesian Coordinates}
\begin{examplebox}
Find the \textbf{rectangular coordinates} of the point \( p \) whose polar coordniates are \( 6, \dfrac{\pi}{3}\).

\begin{solutionbox}
\[
    x = r\cos\theta = c\cos(\dfrac{\pi}{3}) = 6(\frac{1}{2}) = 3
\]
\[
    y = r\sin\theta=6\sin(\dfrac{\pi}{3}) = 6(\frac{\sqrt{3}}{2}) = 3\sqrt{3} 
\]
\[
    \frac{\pi}{3} = 60\text{\textdegree}
\]
Therefore, the cartesian coordinate is \( x, y = (3, 3\sqrt{3}) \).
\end{solutionbox}
\end{examplebox}

\section*{Converting from Cartesian Coordinates to Polar Coordinates}
\begin{examplebox}
Find the polar coordinate of the point \( p \) whose rectangular coordinates are \( -2, 2\sqrt{3} \).

\begin{solutionbox}
Recall that (the circle equation):
\[
    x^2 + y^2 = r^2
\]
It follows that:
\[
    (-2)^2 + (2\sqrt{3})^2 = r^2
\]
\[
    4 + 4 \cdot 3 = r^2
\]
\[
    16 = r^2
\]
\[
    \pm \sqrt{16} = r 
\]
\[
    r = \pm 4
\]
Note that the radius is positive. Thus:
\[
    r = 4 \text{.}
\]
Note that
\end{solutionbox}
\end{examplebox}

\section*{Key Concepts}
\begin{definitionbox}
A \textbf{parametric equation} is a set of equations that express the coordinates of the points of a curve as functions of a variable, called a parameter.
\end{definitionbox}

\section*{Examples}
\begin{examplebox}
\textbf{Example 1:} Consider the parametric equations:
\[ x = t, \quad y = t^2, \quad t \in \mathbb{R}. \]
\begin{itemize}
    \item At $t = 0$, $(x, y) = (0, 0)$.
    \item At $t = 1$, $(x, y) = (1, 1)$.
\end{itemize}
This describes a parabola.

\begin{figure}[H]
    \centering
    \includegraphics[width=0.35\textwidth]{sample_image.jpg}
    \caption{Sample image illustrating the concept.}
    \label{fig:sample_image}
\end{figure}

\end{examplebox}

\section*{Theorems and Proofs}
\begin{theorembox}
\textbf{Theorem:} If $x(t)$ and $y(t)$ are differentiable functions, the slope of the curve is given by:
\[ \frac{dy}{dx} = \frac{\frac{dy}{dt}}{\frac{dx}{dt}}, \quad \text{provided } \frac{dx}{dt} \neq 0. \]

\begin{figure}[H]
    \centering
    \includegraphics[width=0.35\textwidth]{sample_image1.jpg}
    \caption{Graphical representation of the theorem.}
    \label{fig:sample_image1}
\end{figure}

\end{theorembox}

\section*{Additional Notes}
\begin{notebox}
Always check the domain of the parameter $t$ when solving problems involving parametric equations.
\end{notebox}

\end{document}
