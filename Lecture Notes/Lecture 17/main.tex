\documentclass{article}
\usepackage{amsthm,amsmath,amsfonts,lipsum}
\usepackage[T1]{fontenc}
\usepackage{beramono}
\usepackage{listings}
\usepackage{fontawesome5}
\usepackage{adjustbox}
\usepackage{mathabx}
\usepackage{thmtools}
\usepackage{import}
\usepackage{graphicx}
\usepackage{setspace}
\usepackage{geometry}
\usepackage{physics}
\usepackage{float}
\usepackage[english]{babel}
\usepackage{framed}
\usepackage[dvipsnames,x11names]{xcolor}
\usepackage{tcolorbox}
\usepackage{fancyhdr}
\usepackage{hyperref}
\usepackage{booktabs}
\usepackage{enumitem}
\usepackage{cancel}
\usepackage{background}
\usepackage{units}

% Configuring the background
\backgroundsetup{
  scale=1, % Optional, scale if needed
  color=black, % Optional, set the image color, can be omitted
  opacity=0.18, % Optional, adjust opacity for watermark effect
  angle=0,
  position=current page.center, % Center the image on the page
  contents={\includegraphics[width=1.75\paperwidth, height=1.75\paperheight, keepaspectratio]{ninym_ralei_leaf (watermarked by AlexanderTheMango)}} % Keeps aspect ratio and scales to fill the page
}

% Colours
\definecolor{darkgreen}{rgb}{0.0, 0.5, 0.0}
\definecolor{Firebrick}{rgb}{0.698, 0.132, 0.203}
\definecolor{Crimson}{rgb}{0.862745, 0.078431, 0.235294} % Crimson color
\definecolor{lightred}{rgb}{1.0, 0.819608, 0.819608} % Light red for background
\definecolor{MediumPurple}{rgb}{0.576, 0.439, 0.859}
\definecolor{chocolate}{rgb}{0.82, 0.41, 0.12} % Chocolate color definition
% Define the Navy color
\definecolor{Navy}{rgb}{0.0, 0.0, 0.5}

% Define custom tcolorbox styles for notes
\tcbuselibrary{skins, breakable}
\newtcolorbox{definitionbox}{colframe=RoyalBlue, colback=blue!5!white, title=Definition}
\newtcolorbox{examplebox}{colframe=ForestGreen, colback=green!5!white, title=Example}
\newtcolorbox{notebox}{colframe=RedOrange, colback=orange!5!white, title=Note}
\newtcolorbox{theorembox}{colframe=RoyalPurple, colback=purple!5!white, title=Theorem}

\newtcolorbox{propositionbox}{colframe=Goldenrod, colback=yellow!10!white, title=Proposition}
\newtcolorbox{remarkbox}{colframe=MidnightBlue, colback=blue!10!white, title=Remark}
\newtcolorbox{corollarybox}{colframe=OliveGreen, colback=green!10!white, title=Corollary}
\newtcolorbox{warningbox}{colframe=Crimson, colback=lightred, title=Warning}
\newtcolorbox{proofbox}{colframe=Black, colback=gray!10!white, title=Proof}
\newtcolorbox{questionbox}{colframe=Teal, colback=teal!10!white, title=Question}
\newtcolorbox{tipbox}{colframe=Goldenrod, colback=yellow!10!white, title=Tip}
\newtcolorbox{exercisebox}{colframe=darkgreen, colback=green!5!white, title=Exercise}
\newtcolorbox{solutionbox}{colframe=DodgerBlue4, colback=blue!5!white, title=Solution}
\newtcolorbox{algorithmbox}{colframe=Navy, colback=blue!10!white, title=Algorithm}
\newtcolorbox{conceptbox}{colframe=chocolate, colback=brown!10!white, title=Concept}
\newtcolorbox{illustrationbox}{colframe=Firebrick, colback=red!10!white, title=Illustration}
\newtcolorbox{intuitionbox}{colframe=MediumPurple, colback=purple!10!white, title=Intuition}
\newtcolorbox{answerbox}{colframe=RoyalBlue, colback=blue!10!white, title=Answer}

% Geometry settings
\geometry{letterpaper, portrait, includeheadfoot=true, hmargin=1in, vmargin=1in}
\onehalfspacing

% Header and footer
\pagestyle{fancy}
\fancyhf{}
\lhead{MAT232 - Lecture Notes}
\rhead{\thepage}
\lfoot{University of Toronto Mississauga}
\rfoot{\today}

% Document starts
\begin{document}
\renewcommand{\familydefault}{\rmdefault}

\begin{titlepage}
    \null % This is a TeX command that does nothing but is necessary for vfill to work correctly
    \vfill
    \begin{center}
        {\fontsize{40}{48}\selectfont \bfseries MAT232 - Lecture 3}
        \vspace{20pt} \\
        {\LARGE Polar Coordinates and the Arc Length of Parametric Curves} \\
        \vspace{20pt}
        \textbf{AlexanderTheMango}
        \vspace{8pt}
        \\ Prepared for January 13, 2025
    \end{center}
    \vfill
\end{titlepage}


\setcounter{page}{0}
\newpage
\tableofcontents
\newpage

\input{preliminary}



\input{intolecturecontent}
\normalsize

\setcounter{page}{1}

\section*{Reminders}
\addcontentsline{toc}{subsection}{Reminders}

\begin{notebox}
    Term test 3 information:
\end{notebox}
\begin{notebox}
    Final exam information:
\end{notebox}

% add a section here
\begin{examplebox}
    Set up \( \int_{}^{} \int_{}^{} (2x - y^2) dA \) where \( R \) is the triangular region bounded by \( y = -x + 1 \) \( y = x + 1 \), and \( y = 3 \) in both ways (i.e. \( dydx \) and \( dxdy \)).
    \begin{illustrationbox}
        add stuff here lol
    \end{illustrationbox}
    \begin{solutionbox}
        First, we need to find the intersection points of the lines. We have \( -x + 1 = x + 1 \) which gives us \( x = 0 \). So, the intersection points are \( (0, 1) \) and \( (0, 3) \).
        \begin{itemize}
            \item \( dydx \): We have \( x \) going from \( 0 \) to \( 1 \) and \( y \) going from \( -x + 1 \) to \( x + 1 \). So, we have
                  \[ \int_{0}^{1} \int_{-x + 1}^{x + 1} (2x - y^2) dydx \]
            \item \( dxdy \): We have \( y \) going from \( 1 \) to \( 3 \) and \( x \) going from \( -y + 1 \) to \( y - 1 \). So, we have
                  \[ \int_{1}^{3} \int_{-y + 1}^{y - 1} (2x - y^2) dxdy \]
        \end{itemize}
        Next, we may optionally evaluate both integrals to verify that they are equal.
        \subsubsection*{Evaluating \( \int_{0}^{1} \int_{-x + 1}^{x + 1} (2x - y^2) dydx \)}
        We have
        \begin{align*}
            \int_{0}^{1} \int_{-x + 1}^{x + 1} (2x - y^2) dydx & = \int_{0}^{1} \left[ 2xy - \frac{y^3}{3} \right]_{-x + 1}^{x + 1} dx \\
            & = \int_{0}^{1} \left[ 2x(x + 1) - \frac{(x + 1)^3}{3} - 2x(-x + 1) + \frac{(-x + 1)^3}{3} \right] dx \\
            & = \int_{0}^{1} \left[ 2x^2 + 2x - \frac{x^3 + 3x^2 + 3x + 1}{3} + 2x^2 - 2x + \frac{-x^3 + 3x^2 - 3x + 1}{3} \right] dx \\
            & = \int_{0}^{1} \left[ 4x^2 - \frac{2}{3} \right] dx \\
            & = \left[ \frac{4}{3}x^3 - \frac{2}{3}x \right]_{0}^{1} \\
            & = \frac{4}{3} - \frac{2}{3} \\
            & = \frac{2}{3}
        \end{align*}
        \subsubsection*{Evaluating \( \int_{1}^{3} \int_{-y + 1}^{y - 1} (2x - y^2) dxdy \)}
        We have
        \begin{align*}
            \int_{1}^{3} \int_{-y + 1}^{y - 1} (2x - y^2) dxdy & = \int_{1}^{3} \left[ x^2 - y^2x \right]_{-y + 1}^{y - 1} dy \\
            & = \int_{1}^{3} \left[ (y - 1)^2 - y^2(y - 1) - ((-y + 1)^2 - y^2(-y + 1)) \right] dy \\
            & = \int_{1}^{3} \left[ y^2 - 2y + 1 - y^3 + y^2 - y - y^2 + 2y - 1 \right] dy \\
            & = \int_{1}^{3} \left[ -y^3 + 2y \right] dy \\
            & = \left[ -\frac{y^4}{4} + y^2 \right]_{1}^{3} \\
            & = -\frac{81}{4} + 9 - \left( -\frac{1}{4} + 1 \right) \\
            & = -\frac{81}{4} + 9 + \frac{1}{4} - 1 \\
            & = -\frac{80}{4} + 8 \\
            & = -20 + 8 \\
            & = -12
        \end{align*}
    \end{solutionbox}

\end{examplebox}

\begin{examplebox}
    Reverse the order of the integral defined by \( \int_{-1}^{1} \int_{1 + y^2}^{2y^2} (2x + y) dxdy \).

    \begin{solutionbox}
        Consider that there are three integrals in the expression. We also have the following bounds:
        \begin{itemize}
            \item \( -1 \leq x \leq 1 \)
            \item \( 1 + y^2 \leq x \leq 2y^2 \)
        \end{itemize}
        We can reverse the order of the integral by considering the bounds of the integral. We have
        \begin{align*}
            \int_{y = -1}^{y = 1} \int_{x = 1 + y^2}^{x = 2y^2} (2x + y) dxdy \\
            = \int_{x = 0}^{x = 1} \int_{y = -\sqrt{\frac{x}{2}}}^{y = \sqrt{\frac{x}{2}} } (2x + y) dx dy + \int_{x = 1}^{x = 2} \int_{y = -\sqrt{x - 1}}^{y = \sqrt{\frac{x}{2}}} (2x + y) dx dy + \int_{x = 1}^{x = 2} \int_{y = -\sqrt{\frac{x}{2}}}^{y = -\sqrt{x - 1}} (2x + y) dx dy \\
        \end{align*}
    \end{solutionbox}
\end{examplebox}

\begin{examplebox}
    Rewrite the following integral as a single double integral:
    \[
        \int_{0}^{\frac{1}{2}} \int_{0}^{2y} (2 - x - 2y) dxdy + \int_{\frac{1}{2}}^{1} \int_{0}^{2 - 2y} (2 - x - 2y) dxdy
    \]

    \begin{solutionbox}
        \begin{illustrationbox}
            \begin{center}
                \includegraphics[width=0.5\textwidth]{sample_image.jpg}
            \end{center}
        \end{illustrationbox}
        \begin{itemize}
            \item \( x = 2 - 2y \implies y = 1 - \frac{x}{2} \)
        \end{itemize}
        The triangular region is denoted by:
        \begin{itemize}
            \item \( y = \frac{x}{2} \)
            \item \( x = 0 \)
            \item \( y = 1 - \frac{x}{2} \)
        \end{itemize}

        Thus, we consider \( y_{upper} \) and \( y_{lower} \) to find the bounds of the integral. We have
        \begin{align*}
            y_1 &= y_2 \\
            \frac{x}{2} &= 1 - \frac{x}{2} \\
            x &= 1
        \end{align*}

    \end{solutionbox}
\end{examplebox}

\begin{tipbox}
    Soemtimes it is easier to evaluate the integral by reversing the order of integration.
\end{tipbox}

\begin{notebox}
    Sometimes it is the region that is difficult to describe, not the integral itself. In other times, it is the function that has a difficult integral.
\end{notebox}

Some functions are not integrable, such as \( f(x) = \frac{1}{x} \) on \( [0, 1] \).

However, feel free to always use \( u \)-substitution as done in first year calculus.

However, you can reverse the order of integration to actually solve the question from an integral that is not integrable.

\end{document}
