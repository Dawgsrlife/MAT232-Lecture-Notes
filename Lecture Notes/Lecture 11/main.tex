\documentclass{article}
\usepackage{amsthm,amsmath,amsfonts,lipsum}
\usepackage[T1]{fontenc}
\usepackage{beramono}
\usepackage{listings}
\usepackage{fontawesome5}
\usepackage{adjustbox}
\usepackage{mathabx}
\usepackage{thmtools}
\usepackage{import}
\usepackage{graphicx}
\usepackage{setspace}
\usepackage{geometry}
\usepackage{physics}
\usepackage{float}
\usepackage[english]{babel}
\usepackage{framed}
\usepackage[dvipsnames,x11names]{xcolor}
\usepackage{tcolorbox}
\usepackage{fancyhdr}
\usepackage{hyperref}
\usepackage{booktabs}
\usepackage{enumitem}
\usepackage{cancel}
\usepackage{background}
\usepackage{units}

% Configuring the background
\backgroundsetup{
  scale=1, % Optional, scale if needed
  color=black, % Optional, set the image color, can be omitted
  opacity=0.18, % Optional, adjust opacity for watermark effect
  angle=0,
  position=current page.center, % Center the image on the page
  contents={\includegraphics[width=1.75\paperwidth, height=1.75\paperheight, keepaspectratio]{ninym_ralei_leaf (watermarked by AlexanderTheMango)}} % Keeps aspect ratio and scales to fill the page
}

% Colours
\definecolor{darkgreen}{rgb}{0.0, 0.5, 0.0}
\definecolor{Firebrick}{rgb}{0.698, 0.132, 0.203}
\definecolor{Crimson}{rgb}{0.862745, 0.078431, 0.235294} % Crimson color
\definecolor{lightred}{rgb}{1.0, 0.819608, 0.819608} % Light red for background
\definecolor{MediumPurple}{rgb}{0.576, 0.439, 0.859}
\definecolor{chocolate}{rgb}{0.82, 0.41, 0.12} % Chocolate color definition
% Define the Navy color
\definecolor{Navy}{rgb}{0.0, 0.0, 0.5}

% Define custom tcolorbox styles for notes
\tcbuselibrary{skins, breakable}
\newtcolorbox{definitionbox}{colframe=RoyalBlue, colback=blue!5!white, title=Definition}
\newtcolorbox{examplebox}{colframe=ForestGreen, colback=green!5!white, title=Example}
\newtcolorbox{notebox}{colframe=RedOrange, colback=orange!5!white, title=Note}
\newtcolorbox{theorembox}{colframe=RoyalPurple, colback=purple!5!white, title=Theorem}

\newtcolorbox{propositionbox}{colframe=Goldenrod, colback=yellow!10!white, title=Proposition}
\newtcolorbox{remarkbox}{colframe=MidnightBlue, colback=blue!10!white, title=Remark}
\newtcolorbox{corollarybox}{colframe=OliveGreen, colback=green!10!white, title=Corollary}
\newtcolorbox{warningbox}{colframe=Crimson, colback=lightred, title=Warning}
\newtcolorbox{proofbox}{colframe=Black, colback=gray!10!white, title=Proof}
\newtcolorbox{questionbox}{colframe=Teal, colback=teal!10!white, title=Question}
\newtcolorbox{tipbox}{colframe=Goldenrod, colback=yellow!10!white, title=Tip}
\newtcolorbox{exercisebox}{colframe=darkgreen, colback=green!5!white, title=Exercise}
\newtcolorbox{solutionbox}{colframe=DodgerBlue4, colback=blue!5!white, title=Solution}
\newtcolorbox{algorithmbox}{colframe=Navy, colback=blue!10!white, title=Algorithm}
\newtcolorbox{conceptbox}{colframe=chocolate, colback=brown!10!white, title=Concept}
\newtcolorbox{illustrationbox}{colframe=Firebrick, colback=red!10!white, title=Illustration}
\newtcolorbox{intuitionbox}{colframe=MediumPurple, colback=purple!10!white, title=Intuition}
\newtcolorbox{answerbox}{colframe=RoyalBlue, colback=blue!10!white, title=Answer}

% Geometry settings
\geometry{letterpaper, portrait, includeheadfoot=true, hmargin=1in, vmargin=1in}
\onehalfspacing

% Header and footer
\pagestyle{fancy}
\fancyhf{}
\lhead{MAT232 - Lecture Notes}
\rhead{\thepage}
\lfoot{University of Toronto Mississauga}
\rfoot{\today}

% Document starts
\begin{document}
\renewcommand{\familydefault}{\rmdefault}

\begin{titlepage}
    \null % This is a TeX command that does nothing but is necessary for vfill to work correctly
    \vfill
    \begin{center}
        {\fontsize{40}{48}\selectfont \bfseries MAT232 - Lecture 3}
        \vspace{20pt} \\
        {\LARGE Polar Coordinates and the Arc Length of Parametric Curves} \\
        \vspace{20pt}
        \textbf{AlexanderTheMango}
        \vspace{8pt}
        \\ Prepared for January 13, 2025
    \end{center}
    \vfill
\end{titlepage}


\setcounter{page}{0}
\newpage
\tableofcontents
\newpage

\input{preliminary}



\input{intolecturecontent}
\normalsize

\setcounter{page}{1}

\subsection*{Review of Last Lecture}
\addcontentsline{toc}{subsection}{Review of Last Lecture}

(stuff goes here)

\subsection*{Section 4.4: Tangent Planes}
\addcontentsline{toc}{subsection}{Section 4.4: Tangent Planes}

\subsubsection*{Recall from 1st Year Calculus}
\addcontentsline{toc}{subsubsection}{Recall from 1st Year Calculus}

\begin{definitionbox}
    Tangent lines are denoted by:
    \[
        y = f(x) \quad \text{at} \quad x = x_0 \text{ (given)}
    \]
    \begin{itemize}
        \item Point $P = (x_0, f(x_0)) = (x_0, y_0)$
        \item Slope of tangent line: $m = f'(x_0)$
        \begin{itemize}
            \item So at $x = x_0$, the slope of the tangent line is $f'(x_0) = m$
        \end{itemize}
    \end{itemize}

    \begin{align*}
        y - y_0 &= m(x - x_0) \\
        y = f'(x_0)(x - x_0) + f(x_0) \\
        y = f(x_0) + f'(x_0)(x - x_0)
    \end{align*}

    \begin{remarkbox}
        \textbf{Note:} The tangent line is a linear approximation of the function $f(x)$ near $x = x_0$. \\
        We will not be using this formula in this course, but it is good to know.
    \end{remarkbox}
\end{definitionbox}

\subsubsection*{Now, in MAT232}
\addcontentsline{toc}{subsubsection}{Now, in MAT232}

\begin{definitionbox}
    Plane Equation:
    \[
        a(x - x_0) + b(y - y_0) + c(z - z_0) = 0
    \]
    is the equation of a plane in $\mathbb{R}^3$ in \textbf{point-normal or scalar form}.
    Rearrange and notice:
    \begin{align*}
        z - z_0 &= -\frac{a}{c}(x - x_0) - \frac{b}{c}(y - y_0) \\
    \end{align*}
    where \( z = f(x, y) \) and \( z_0 = f(x_0, y_0) \).

    So, \( z_0 = f(x_0, y_0) \) is the point on the surface of the function \( z = f(x, y) \) at \( (x_0, y_0) \), and \( -\frac{a}{c} = f_x(x_0, y_0) \) and \( -\frac{b}{c} = f_y(x_0, y_0) \) are the partial derivatives of \( f(x, y) \) at \( (x_0, y_0) \).
    This form is called the \textbf{tangent plane} to the surface of the function \( z = f(x, y) \) at \( (x_0, y_0) \).
    \begin{notebox}
    This equation will be included on the formula sheet.
    \end{notebox}
\end{definitionbox}

\begin{conceptbox}
Set \( y = y_0 \):
\begin{align*}
    z = f(x_0, y_0) + f_x(x_0, y_0)(x - x_0) + f_y(x_0, y_0)(y - y_0) \\
    z = f(x_0, y_0) + f_x(x_0, y_0)(x - x_0) + \cancelto{0}{f_y(x_0, y_0)(y - y_0)}
\end{align*}
Call this \( T_1 \), the tangent plane to the surface of the function \( z = f(x, y) \) at \( (x_0, y_0) \) when \( y = y_0 \).
\bigskip
Set \( x = x_0 \):
\begin{align*}
    z = f(x_0, y_0) + f_x(x_0, y_0)(x - x_0) + f_y(x_0, y_0)(y - y_0) \\
    z = f(x_0, y_0) + \cancelto{0}{f_x(x_0, y_0)(x - x_0)} + f_y(x_0, y_0)(y - y_0)
\end{align*}
Call this \( T_2 \), the tangent plane to the surface of the function \( z = f(x, y) \) at \( (x_0, y_0) \) when \( x = x_0 \).
\end{conceptbox}

\subsection*{Let's Try an Example}
\addcontentsline{toc}{subsection}{Let's Try an Example}

\begin{examplebox}
    Find the equation of the tangent plane to the surface \( z = f(x, y) = \ln(x - 2y) \) at the point \( (x_0, y_0) = (3, 1) \).

    \begin{solutionbox}
    Tangent plane equation:
    \[
        z = f(x_0, y_0) + f_x(x_0, y_0)(x - x_0) + f_y(x_0, y_0)(y - y_0)
    \]
    \begin{itemize}
        \item Point: \( (x_0, y_0) = (3, 1) \)
    \end{itemize}
    
    So\dots
    \begin{align*}
        z_0 = f(x_0, y_0) &= \ln(3 - 2(1)) = \ln(3 - 2) = \ln(1) = 0 \\
        f_x(x, y) &= \frac{1}{x - 2y} \\
        f_y(x, y) &= \frac{-2}{x - 2y}
    \end{align*}
    \begin{itemize}
        \item Partial derivatives at \( (x_0, y_0) = (3, 1) \):
        \begin{align*}
            f_x(3, 1) &= \frac{1}{3 - 2(1)} = \frac{1}{3 - 2} = 1 \\
            f_y(3, 1) &= \frac{-2}{3 - 2(1)} = \frac{-2}{3 - 2} = -2
        \end{align*}
    \end{itemize}
    So, the equation of the tangent plane is:
    \begin{align*}
        z &= 0 + 1(x - 3) - 2(y - 1) \\
        z &= x - 3 - 2y + 2 \\
        z &= x - 2y - 1
    \end{align*}
    \begin{answerbox}
        The equation of the tangent plane to the surface \( z = f(x, y) = \ln(x - 2y) \) at the point \( (x_0, y_0) = (3, 1) \) is \( z = x - 2y - 1 \).
    \end{answerbox}
\end{solutionbox}
\end{examplebox}

\subsection*{Another Example}
\addcontentsline{toc}{subsection}{Another Example}

\begin{examplebox}
    Find the equation of the tangent plane to the surface \( z = f(x, y) = x^2 + y^2 + 1 \) at the point \( (x_0, y_0) = (2, 1) \).

    \begin{solutionbox}
    Tangent plane equation:
    \[
        z = f(x_0, y_0) + f_x(x_0, y_0)(x - x_0) + f_y(x_0, y_0)(y - y_0)
    \]
    \begin{itemize}
        \item Point: \( (x_0, y_0) = (2, 1) \)
    \end{itemize}
    Partial derivatives:
    \begin{align*}
        f_x(x, y) &= 2x \\
        f_y(x, y) &= 2y
    \end{align*}
    Point:
    \begin{align*}
        z_0 = f(x_0, y_0) &= 2^2 + 1^2 + 1 = 4 + 1 + 1 = 6
    \end{align*}
    Partial derivatives at \( (x_0, y_0) = (2, 1) \):
    \begin{align*}
        f_x(2, 1) &= 2(2) = 4 \\
        f_y(2, 1) &= 2(1) = 2
    \end{align*}
    So, the equation of the tangent plane is:
    \begin{align*}
        z &= 6 + 4(x - 2) + 2(y - 1) \\
        z &= 6 + 4x - 8 + 2y - 2 \\
        z &= 4x + 2y - 4
    \end{align*}
    \begin{answerbox}
        The equation of the tangent plane to the surface \( z = f(x, y) = x^2 + y^2 + 1 \) at the point \( (x_0, y_0) = (2, 1) \) is \( z = 4x + 2y - 4 \).
    \end{answerbox}
    \end{solutionbox}
\end{examplebox}

\subsection*{Section 4.5: Chain Rule}
\addcontentsline{toc}{subsection}{Section 4.5: Chain Rule}

\subsubsection*{Recall from 1st Year Calculus}
\addcontentsline{toc}{subsubsection}{Recall from 1st Year Calculus}
Let \( y = f(u) \) and \( u = g(x) \). Then, \( y = f(g(x)) \).
\begin{itemize}
    \item Chain Rule: \( \dv{y}{x} = \dv{y}{u} \cdot \dv{u}{x} \)
\end{itemize}

\begin{examplebox}
    Let \( f(x) = \cos(x) \) and \( g(x) = e^x \). Find \( h(x) = f(g(x)) \) and \( h'(x) \).

    \begin{solutionbox}
        Notice that \( h(x) = f(g(x)) = \cos(e^x) \).
        So \( h'(x) = -\sin(e^x) \cdot e^x \).
        \begin{answerbox}
            \( h(x) = \cos(e^x) \) and \( h'(x) = -\sin(e^x) \cdot e^x \).
        \end{answerbox}
    \end{solutionbox}
\end{examplebox}

\subsection*{Now, in MAT232}
\addcontentsline{toc}{subsection}{Now, in MAT232}

\begin{definitionbox}
    Given \( w = f(x, y) \), \( x = h(t) \), \( y = g(t) \), and that they are all differentiable functions, then \( w = f(x, y) = f(h(t), g(t)) \).
    \begin{conceptbox}
        Chain Rule:
        \[
            \dv{w}{t} = \pdv{w}{x} \cdot \dv{x}{t} + \pdv{w}{y} \cdot \dv{y}{t}
        \]
    \end{conceptbox}
    \begin{notebox}
        This is the chain rule in MAT232.
    \end{notebox}
\end{definitionbox}

Let's illustrate how \( w = f(x, y) \) breaks down:
\begin{align*}
    w = f(x, y) = f(h(t), g(t))
\end{align*}
Note that if \( w = f(x, y) \) depends on \( x \) and \( y \), and \( x \) and \( y \) depend on \( t \), then \( w \) depends on \( t \).

\begin{align*}
    w &= f(x, y) = f(h(t), g(t)) \\
    \pdv{w}{x} &= \pdv{f}{x} = f_x(h(t), g(t)) \cdot h'(t) \\
    \pdv{w}{y} &= \pdv{f}{y} = f_y(h(t), g(t)) \cdot g'(t)
\end{align*}

\begin{examplebox}
    Find the derivative of \( w = f(x, y) = xy \) with respect to \( t \) if \( x = \cos(t) \) and \( y = \sin(t) \) at \( t = \frac{\pi}{2} \).

    Approach 1 (Direct Substitution):
    \begin{solutionbox}
        \begin{align*}
            w &= f(x, y) = xy \\
            &= \cos(t) \cdot \sin(t) \\
            &= \frac{1}{2} \sin(2t)
        \end{align*}
        \begin{answerbox}
            The derivative of \( w = f(x, y) = xy \) with respect to \( t \) if \( x = \cos(t) \) and \( y = \sin(t) \) at \( t = \frac{\pi}{2} \) is \( \cos(2t) \).
        \end{answerbox}
    \end{solutionbox}

    Approach 1 (Substitute \( t = \frac{\pi}{2} \)):
    \begin{solutionbox}
        \begin{align*}
            w &= f(x, y) = xy \\
            \dv{w}{t} &= \dv{w}{x} \cdot \dv{x}{t} + \dv{w}{y} \cdot \dv{y}{t} \\
            &= y \cdot (-\sin(t)) + x \cdot \cos(t) \\
            &= \sin(t) \cdot (-\sin(t)) + \cos(t) \cdot \cos(t) \\
            &= \sin\left(\frac{\pi}{2}\right) \cdot (-\sin\left(\frac{\pi}{2}\right)) + \cos\left(\frac{\pi}{2}\right) \cdot \cos\left(\frac{\pi}{2}\right) \\
            &= 1 \cdot (-1) + 0 \cdot 0 = -1
        \end{align*}
        \begin{answerbox}
            The derivative of \( w = f(x, y) = xy \) with respect to \( t \) if \( x = \cos(t) \) and \( y = \sin(t) \) at \( t = \frac{\pi}{2} \) is \( -1 \).
        \end{answerbox}
    \end{solutionbox}
\end{examplebox}

\end{document}
