\documentclass{article}
\usepackage{amsthm,amsmath,amsfonts,lipsum}
\usepackage[T1]{fontenc}
\usepackage{beramono}
\usepackage{listings}
\usepackage{fontawesome5}
\usepackage{adjustbox}
\usepackage{mathabx}
\usepackage{thmtools}
\usepackage{import}
\usepackage{graphicx}
\usepackage{setspace}
\usepackage{geometry}
\usepackage{physics}
\usepackage{float}
\usepackage[english]{babel}
\usepackage{framed}
\usepackage[dvipsnames,x11names]{xcolor}
\usepackage{tcolorbox}
\usepackage{fancyhdr}
\usepackage{hyperref}
\usepackage{booktabs}
\usepackage{enumitem}
\usepackage{cancel}
\usepackage{background}
\usepackage{units}

% Configuring the background
\backgroundsetup{
  scale=1, % Optional, scale if needed
  color=black, % Optional, set the image color, can be omitted
  opacity=0.18, % Optional, adjust opacity for watermark effect
  angle=0,
  position=current page.center, % Center the image on the page
  contents={\includegraphics[width=1.75\paperwidth, height=1.75\paperheight, keepaspectratio]{ninym_ralei_leaf (watermarked by AlexanderTheMango)}} % Keeps aspect ratio and scales to fill the page
}

% Colours
\definecolor{darkgreen}{rgb}{0.0, 0.5, 0.0}
\definecolor{Firebrick}{rgb}{0.698, 0.132, 0.203}
\definecolor{Crimson}{rgb}{0.862745, 0.078431, 0.235294} % Crimson color
\definecolor{lightred}{rgb}{1.0, 0.819608, 0.819608} % Light red for background
\definecolor{MediumPurple}{rgb}{0.576, 0.439, 0.859}
\definecolor{chocolate}{rgb}{0.82, 0.41, 0.12} % Chocolate color definition

% Define custom tcolorbox styles for notes
\tcbuselibrary{skins, breakable}
\newtcolorbox{definitionbox}{colframe=RoyalBlue, colback=blue!5!white, title=Definition}
\newtcolorbox{examplebox}{colframe=ForestGreen, colback=green!5!white, title=Example}
\newtcolorbox{notebox}{colframe=RedOrange, colback=orange!5!white, title=Note}
\newtcolorbox{theorembox}{colframe=RoyalPurple, colback=purple!5!white, title=Theorem}

\newtcolorbox{propositionbox}{colframe=Goldenrod, colback=yellow!10!white, title=Proposition}
\newtcolorbox{remarkbox}{colframe=MidnightBlue, colback=blue!10!white, title=Remark}
\newtcolorbox{corollarybox}{colframe=OliveGreen, colback=green!10!white, title=Corollary}
\newtcolorbox{warningbox}{colframe=Crimson, colback=lightred, title=Warning}
\newtcolorbox{proofbox}{colframe=Black, colback=gray!10!white, title=Proof}
\newtcolorbox{questionbox}{colframe=Teal, colback=teal!10!white, title=Question}
\newtcolorbox{tipbox}{colframe=Goldenrod, colback=yellow!10!white, title=Tip}
\newtcolorbox{exercisebox}{colframe=darkgreen, colback=green!5!white, title=Exercise}
\newtcolorbox{solutionbox}{colframe=DodgerBlue4, colback=blue!5!white, title=Solution}
\newtcolorbox{algorithmbox}{colframe=Navy, colback=blue!10!white, title=Algorithm}
\newtcolorbox{conceptbox}{colframe=chocolate, colback=brown!10!white, title=Concept}
\newtcolorbox{illustrationbox}{colframe=Firebrick, colback=red!10!white, title=Illustration}
\newtcolorbox{intuitionbox}{colframe=MediumPurple, colback=purple!10!white, title=Intuition}
\newtcolorbox{answerbox}{colframe=RoyalBlue, colback=blue!10!white, title=Answer}

% Geometry settings
\geometry{letterpaper, portrait, includeheadfoot=true, hmargin=1in, vmargin=1in}
\onehalfspacing

% Header and footer
\pagestyle{fancy}
\fancyhf{}
\lhead{MAT232 - Lecture Notes}
\rhead{\thepage}
\lfoot{University of Toronto Mississauga}
\rfoot{\today}

% Document starts
\begin{document}
\renewcommand{\familydefault}{\rmdefault}

\begin{titlepage}
    \null % This is a TeX command that does nothing but is necessary for vfill to work correctly
    \vfill
    \begin{center}
        {\fontsize{40}{48}\selectfont \bfseries MAT232 - Lecture 3}
        \vspace{20pt} \\
        {\LARGE Polar Coordinates and the Arc Length of Parametric Curves} \\
        \vspace{20pt}
        \textbf{AlexanderTheMango}
        \vspace{8pt}
        \\ Prepared for January 13, 2025
    \end{center}
    \vfill
\end{titlepage}

\input{preliminary}

\section*{Preliminary Definitions and Theorems}

% Definition: Polar Coordinates
\begin{definitionbox}
\textbf{Polar Coordinates.} \\
Each point in the Cartesian plane can be represented in polar coordinates as an ordered pair $(r, \theta)$, where $r$ is the radial coordinate (distance from the origin), and $\theta$ is the angular coordinate (angle measured from the positive $x$-axis).  
The correspondence between Cartesian coordinates $(x, y)$ and polar coordinates $(r, \theta)$ is given by:
\[
x = r\cos\theta, \quad y = r\sin\theta, \quad r^2 = x^2 + y^2, \quad \tan\theta = \frac{y}{x}.
\]
\end{definitionbox}

% Theorem: Converting Points
\begin{theorembox}
\textbf{Theorem 1.4. Converting Points Between Coordinate Systems.} \\
Given a point $P$ in the plane with Cartesian coordinates $(x, y)$ and polar coordinates $(r, \theta)$, the following conversion formulas hold true:
\[
x = r\cos\theta, \quad y = r\sin\theta,
\]
\[
r^2 = x^2 + y^2, \quad \tan\theta = \frac{y}{x}.
\]
These formulas can be used to convert between Cartesian and polar coordinates.
\end{theorembox}

% Example: Converting Coordinates
\begin{examplebox}
\textbf{Example 1.10. Converting Between Rectangular and Polar Coordinates.}
\begin{enumerate}
    \item Convert $(1, 1)$ to polar coordinates:  
    Use $x = 1$ and $y = 1$. Then:
    \[
    r^2 = x^2 + y^2 = 1^2 + 1^2 = 2 \implies r = \sqrt{2}, \quad \tan\theta = \frac{y}{x} = \frac{1}{1} = 1 \implies \theta = \frac{\pi}{4}.
    \]
    Therefore, $(1, 1)$ can be represented as $(\sqrt{2}, \dfrac{\pi}{4})$ in polar coordinates.
\end{enumerate}
\end{examplebox}

% Problem-Solving Strategy: Plotting a Curve
\begin{conceptbox}
\textbf{Problem-Solving Strategy: Plotting a Curve in Polar Coordinates.}
\begin{enumerate}
    \item Create a table with two columns: one for $\theta$ values and one for $r$ values.
    \item Calculate the corresponding $r$ values for each $\theta$.
    \item Plot each ordered pair $(r, \theta)$ on the polar coordinate axes.
    \item Connect the points and observe the resulting graph.
\end{enumerate}
\end{conceptbox}

% Example: Graphing in Polar Coordinates
\begin{examplebox}
\textbf{Example 1.12. Graphing a Function in Polar Coordinates.} \\
Graph the curve defined by $r = 4\sin\theta$.  
\begin{enumerate}
    \item Create a table of values for $\theta$ and calculate $r$:
    \[
    \begin{array}{c|c|c|c|c|c|c}
    \theta & 0 & \frac{\pi}{6} & \frac{\pi}{4} & \frac{\pi}{2} & \pi & 2\pi \\
    \hline
    r = 4\sin\theta & 0 & 2 & 2\sqrt{2} & 4 & 0 & 0
    \end{array}
    \]
    \item Plot the points and connect them to form the curve. The result is a circle with radius $2$ centered at $(0, 2)$ in rectangular coordinates.
\end{enumerate}
\begin{figure}[H]
    \centering
    \includegraphics[width=0.5\textwidth]{rEquals4sinTheta.jpg}
    \caption{The graph of the function \( r = 4\sin\theta \) is a circle.}
    \label{fig:sample_image}
\end{figure}
\end{examplebox}

\input{intolecturecontent}
\normalsize

\section*{Recall: 1st Year Calculus}
\begin{definitionbox}
The \textbf{definite integral} of a function \( y = f(x) \), where \( f(x) \geq 0 \), represents the area under the curve from \( x = a \) to \( x = b \):
\[
    \text{Area} = \int_{x=a}^{x=b} f(x) \, dx
\]
\begin{figure}[H]
    \centering
    \includegraphics[width=0.6\textwidth]{1styearcalc.jpg}
    \caption{Illustration of the area under \( y = f(x) \).}
    \label{fig:sample_image}
\end{figure}
\end{definitionbox}

\section*{Section 1.2: MAT232 Perspective}
\begin{definitionbox}
A \textbf{parametric curve} is defined by:
\[
    x = f(t), \quad y = g(t), \quad \alpha \leq t \leq \beta
\]
with the following properties:
\begin{itemize}
    \item The curve lies above the \( x \)-axis.
    \item The curve does not self-intersect.
\end{itemize}
\begin{figure}[H]
    \centering
    \includegraphics[width=0.25\textwidth]{self_intersecting_parametric_curve_example.png}
    \caption{A curve that self-intersects.}
    \label{fig:sample_image}
\end{figure}

The area enclosed by the curve is given by:
\[
    \text{Area} = \int_{t=\alpha}^{t=\beta} g(t) f'(t) \, dt
\]
or equivalently:
\[
    \text{Area} = \int_{t=\alpha}^{t=\beta} f(t) g'(t) \, dt
\]
\end{definitionbox}

\subsection*{Alternative Forms for Area}
\begin{notebox}
In specific cases, the area can also be calculated using:
\[
    \text{Area} = \int_{y=c}^{y=d} x(y) \, dy
\]
or:
\[
    \text{Area} = \int_{x=a}^{x=b} y(x) \, dx
\]
\end{notebox}

\section*{Area Enclosed by a Parametric Curve}
\begin{examplebox}
Calculate the area enclosed by the parametric curve:
\[
    x = \cos(t), \quad y = \sin(t), \quad 0 \leq t \leq \pi
\]

\begin{solutionbox}
The area is calculated as:
\[
    \text{Area} = \int_{t=\alpha}^{t=\beta} g(t) f'(t) \, dt,
\]
where \( x = f(t) \) and \( y = g(t) \). Here, \( f(t) = \cos(t) \), \( g(t) = \sin(t) \), and \( f'(t) = -\sin(t) \). Substituting:
\[
    \text{Area} = \int_{t=0}^{t=\pi} \sin(t)(-\sin(t)) \, dt = \int_{t=0}^{t=\pi} -\sin^2(t) \, dt.
\]

Using \( \sin^2(t) = \frac{1}{2}(1 - \cos(2t)) \), we get:
\[
    \text{Area} = -\int_{t=0}^{t=\pi} \frac{1}{2}(1 - \cos(2t)) \, dt = -\frac{1}{2} \left[ \int_{t=0}^{t=\pi} 1 \, dt - \int_{t=0}^{t=\pi} \cos(2t) \, dt \right].
\]

Evaluate the integrals:
\[
    \int_{t=0}^{t=\pi} 1 \, dt = \pi, \quad \int_{t=0}^{t=\pi} \cos(2t) \, dt = \left[ \frac{\sin(2t)}{2} \right]_0^\pi = 0.
\]

Thus:
\[
    \text{Area} = -\frac{1}{2}(\pi - 0) = -\frac{\pi}{2}.
\]
Taking the absolute value (since area is positive):
\[
    \text{Area} = \frac{\pi}{2}.
\]
\end{solutionbox}
\end{examplebox}

\section*{Area Under the Curve of a Cycloid}
\begin{examplebox}
\textbf{Example:} Find the area under the cycloid defined by:
\[
    x = t - \sin(t), \quad y = 1 - \cos(t), \quad 0 \leq t \leq 2\pi.
\]

\begin{solutionbox}
The area under a parametric curve is given by:
\[
    \text{Area} = \int_{t=\alpha}^{t=\beta} g(t) f'(t) \, dt, \quad f'(t) = \frac{dx}{dt}.
\]
\\
\textbf{Step 1: Substitution} \\
From \( x = t - \sin(t) \) and \( y = 1 - \cos(t) \):  
\[
    f'(t) = 1 - \cos(t), \quad g(t) = 1 - \cos(t).
\]
Substitute into the formula:
\[
    \text{Area} = \int_{0}^{2\pi} (1 - \cos(t))^2 \, dt.
\]
\\
\textbf{Step 2: Expand and Separate Terms} \\
Expand \( (1 - \cos(t))^2 \):
\[
    \text{Area} = \int_{0}^{2\pi} [1 - 2\cos(t) + \cos^2(t)] \, dt.
\]
Split the integral:
\[
    \text{Area} = \int_{0}^{2\pi} 1 \, dt - 2\int_{0}^{2\pi} \cos(t) \, dt + \int_{0}^{2\pi} \cos^2(t) \, dt.
\]
\textit{...cont'd...}
\end{solutionbox}
\end{examplebox}
\begin{examplebox}
\begin{solutionbox}
\textit{...cont'd...} \\
\\
\textbf{Step 3: Evaluate Each Term} \\
1. \underline{First Term:}  
\[
    \int_{0}^{2\pi} 1 \, dt = 2\pi.
\]  
2. \underline{Second Term:}  
\[
    \int_{0}^{2\pi} \cos(t) \, dt = [\sin(t)]_{0}^{2\pi} = 0.
\]  
3. \underline{Third Term:} \\
\\
Using \( \cos^2(t) = \dfrac{1 + \cos(2t)}{2} \):  
\[
    \int_{0}^{2\pi} \cos^2(t) \, dt = \frac{1}{2} \int_{0}^{2\pi} 1 \, dt + \frac{1}{2} \int_{0}^{2\pi} \cos(2t) \, dt.
\]
Evaluate:  
\[
    \frac{1}{2} \int_{0}^{2\pi} 1 \, dt = \pi, \quad \frac{1}{2} \int_{0}^{2\pi} \cos(2t) \, dt = 0.
\]
Thus:
\[
    \int_{0}^{2\pi} \cos^2(t) \, dt = \pi.
\]

\textbf{Step 4: Combine Results}  
\[
    \text{Area} = 2\pi - 0 + \pi = 3\pi.
\]

\begin{answerbox}
\( \text{Area} = 3\pi \) 
\end{answerbox}

\end{solutionbox}
\end{examplebox}


\section*{Homework Practice Question: Area Under a Parametric Curve}
\begin{exercisebox}
    Find the area under the curve defined by
    \[
        x = 3\cos(t) + \cos(3t), \quad y = 3\sin(t) - \sin(3t), \quad 0 \leq t \leq \pi.
    \]
    Hint: Recall that \( \sin^2(x) + \cos^2(x) = 1 \).
    
    \begin{solutionbox}
    The area under a parametric curve is given by:
    \[
        \text{Area} = \int_{t=\alpha}^{t=\beta} g(t) f'(t) \, dt,
    \]
    where \( x = f(t) \), \( y = g(t) \), and \( f'(t) = \frac{dx}{dt} \). \\
    \\
    \textbf{Step 1: Differentiate \( x(t) \)} \\
    Given \( x = 3\cos(t) + \cos(3t) \), compute:
    \[
        f'(t) = \frac{d}{dt}[3\cos(t) + \cos(3t)] = -3\sin(t) - 3\sin(3t).
    \]
    \\
    \textbf{Step 2: Substitute into the Formula} \\
    The parametric area formula becomes:
    \[
        \text{Area} = \int_{0}^{\pi} \big[3\sin(t) - \sin(3t)\big] \big[-3\sin(t) - 3\sin(3t)\big] \, dt.
    \]
    \\
    \textbf{Step 3: Simplify the Expression} \\
    Expand the product:
    \[
        \big[3\sin(t) - \sin(3t)\big]\big[-3\sin(t) - 3\sin(3t)\big] = -9\sin^2(t) - 9\sin(t)\sin(3t) + 3\sin(3t)\sin(t) + 3\sin^2(3t).
    \]
    Combine terms:
    \[
        -9\sin^2(t) + 3\sin^2(3t) - 6\sin(t)\sin(3t).
    \]
    \textit{...cont'd...}
    \end{solutionbox}
\end{exercisebox}
\begin{exercisebox}
    \begin{solutionbox}
    \textit{...cont'd..} \\

    Using the product-to-sum identity for \( \sin(a)\sin(b) = \frac{1}{2}[\cos(a-b) - \cos(a+b)] \):
    \[
        \sin(t)\sin(3t) = \frac{1}{2}[\cos(2t) - \cos(4t)].
    \]
        
    Substitute this back:
    \[
        \text{Area} = \int_{0}^{\pi} \big[-9\sin^2(t) + 3\sin^2(3t) - 3[\cos(2t) - \cos(4t)]\big] \, dt.
    \]
    \\
    \textbf{Step 4: Break the Integral into Separate Terms} \\
    Split the integral:
    \[
        \text{Area} = -9\int_{0}^{\pi} \sin^2(t) \, dt + 3\int_{0}^{\pi} \sin^2(3t) \, dt - 3\int_{0}^{\pi} \cos(2t) \, dt + 3\int_{0}^{\pi} \cos(4t) \, dt.
    \]

    \textbf{Step 5: Evaluate Each Integral}  

    1. \underline{First Term (\(-9\int_{0}^{\pi} \sin^2(t) \, dt\)):} \\
    \\
       Use the identity \( \sin^2(t) = \dfrac{1 - \cos(2t)}{2} \):
       \[
           \int_{0}^{\pi} \sin^2(t) \, dt = \int_{0}^{\pi} \frac{1 - \cos(2t)}{2} \, dt = \frac{1}{2} \int_{0}^{\pi} 1 \, dt - \frac{1}{2} \int_{0}^{\pi} \cos(2t) \, dt.
       \]
       Evaluate:
       \[
           \frac{1}{2} \int_{0}^{\pi} 1 \, dt = \frac{\pi}{2}, \quad \frac{1}{2} \int_{0}^{\pi} \cos(2t) \, dt = \frac{1}{2}[0] = 0.
       \]
       So:
       \[
           \int_{0}^{\pi} \sin^2(t) \, dt = \frac{\pi}{2}.
       \]
       Multiply by \(-9\):
       \[
           -9\int_{0}^{\pi} \sin^2(t) \, dt = -9 \cdot \frac{\pi}{2} = -\frac{9\pi}{2}.
       \]
    \textit{...cont'd...}
    \end{solutionbox}
\end{exercisebox}
\begin{exercisebox}
    \begin{solutionbox}
        \textit{...cont'd...} \\
        \\
        2. \underline{Second Term (\(3\int_{0}^{\pi} \sin^2(3t) \, dt\)):} \\
        \\
       Similarly, \( \sin^2(3t) = \dfrac{1 - \cos(6t)}{2} \):
       \[
           \int_{0}^{\pi} \sin^2(3t) \, dt = \frac{1}{2} \int_{0}^{\pi} 1 \, dt - \frac{1}{2} \int_{0}^{\pi} \cos(6t) \, dt.
       \]
       Evaluate:
       \[
           \frac{1}{2} \int_{0}^{\pi} 1 \, dt = \frac{\pi}{2}, \quad \frac{1}{2} \int_{0}^{\pi} \cos(6t) \, dt = 0.
       \]
       So:
       \[
           \int_{0}^{\pi} \sin^2(3t) \, dt = \frac{\pi}{2}.
       \]
       Multiply by 3:
       \[
           3\int_{0}^{\pi} \sin^2(3t) \, dt = 3 \cdot \frac{\pi}{2} = \frac{3\pi}{2}.
       \]

    3. \underline{Third Term (\(-3\int_{0}^{\pi} \cos(2t) \, dt\)):} \\
    \\
       Since \(\int_{0}^{\pi} \cos(2t) \, dt = 0\):
       \[
           -3\int_{0}^{\pi} \cos(2t) \, dt = 0.
       \]

    4. \underline{Fourth Term (\(3\int_{0}^{\pi} \cos(4t) \, dt\)):} \\
    \\
       Similarly, \(\int_{0}^{\pi} \cos(4t) \, dt = 0\):
       \[
           3\int_{0}^{\pi} \cos(4t) \, dt = 0.
       \]
    \\
    \textbf{Step 6: Combine Results} \\
    Add the evaluated terms:
    \[
        \text{Area} = -\frac{9\pi}{2} + \frac{3\pi}{2} + 0 + 0 = -\frac{6\pi}{2} = -3\pi.
    \]
    However, the area is always positive, so:
    \[
        \text{Area} = 3\pi.
    \]
    \end{solutionbox}
\end{exercisebox}
\begin{exercisebox}
    \begin{solutionbox}
        \textit{...cont'd...}
        \begin{answerbox}
            \( \text{Area} = 3\pi \)
        \end{answerbox}
    \end{solutionbox}
\end{exercisebox}

\textbf{continue here with embellishments!}
\section*{The Arc Length of a Parametric Curve}
\begin{theorembox}
\textbf{Theorem:} \textbf{self-note: grab the actual theorem from the textbook lol}

\begin{itemize}
    \item \( (x_1, y_1) \) and \( (x_2, y_2) \) are points
    \item \( \Delta x = x_1 - x_2 \), \( \Delta = \text{Delta} \) 
\end{itemize}
The distance between two points is denoted by \( D \) as follows:
\[
    D = \sqrt{(x_1 - x_2)^2 + (y_1 - y_2)^2}
\]
Substitute \( \Delta x \) and \( \Delta y \) as follows:
\[
    D = \sqrt{\Delta x^2 + \Delta y^2} 
\]
It follows that\dots
\[
    D = \sqrt{(\frac{\Delta x}{\Delta t})^2 + (\frac{\Delta y}{\Delta t})^2 } \Delta t
\]
Now, notice the similarity to Riemann sums from MAT136. As \( \Delta x \to 0 \):
\[
    L = \int_{t=\alpha}^{t=\beta} \sqrt{(\frac{dx}{dt})^2 + (\frac{dy}{dt})^2} dt 
\]
\textit{\( L \) is the (arc) length of a curve. This is confirmed to be included on term test 1, and will be on the formula sheet.}

\end{theorembox}

\begin{figure}[H]
    \centering
    \includegraphics[width=0.6\textwidth]{sample_image1.jpg}
    \caption{Graphical representation of the theorem.}
    \label{fig:sample_image1}
\end{figure}

\section*{Example}
\begin{examplebox}
    Find the arc length of the curve defined by
    \[
        x = 3\cos(t), \quad y = 3\sin(t), \quad t \in [0, 2\pi] \text{.}
    \]
    The arc length is denoted by \( L \). Evaluate as follows:
    \begin{equation*}
        \begin{aligned}
            L &= \int_{0}^{2\pi} \sqrt{(-3\sin(t))^2 + (3\cos(t))^2} dt \\
            &= \textbf{self-note: finish this using the notes in the camera roll}
        \end{aligned}
    \end{equation*}
\end{examplebox}

\section*{Homework Practice Problem}
\begin{notebox}
    Find the arc length of the curve defined by
    \[
        x = 3t^2, \quad y = 2t^3, \quad 1 \leq t \leq 3 \text{.}
    \]
    \textbf{self-note: do the solution to this} \\
    \\
\end{notebox}

\section*{Section 1.3: Polar Coordinates}
\begin{definitionbox}
    \textbf{give the actual definition here from the textbook lol} \\
    \\
    Cartesian Coordniates:
    \begin{figure}[H]
        \centering
        \includegraphics[width=0.6\textwidth]{sample_image1.jpg}
        \caption{Graphical representation of the theorem.}
        \label{fig:sample_image1}
    \end{figure}
    Polar Coordinates:
    \begin{figure}[H]
        \centering
        \includegraphics[width=0.6\textwidth]{sample_image1.jpg}
        \caption{Graphical representation of the theorem.}
        \label{fig:sample_image1}
    \end{figure}
    How to work with \textbf{Polar Coordinates}
    \begin{enumerate}
        \item Start from the origin, and trace positively along the x-axis by the amount of the radius
        \item Imagine a line segment connecting from the origin to the resultant point, and rotate this line segment \( \theta \) degrees about the origin
        \item The destination of the resultant point is the point represented by the polar coordinate \( (r, \theta) \).
    \end{enumerate}
    Converting from \textbf{Cartesian Coordinates} to \textbf{Polar Coordinates}:
    \begin{enumerate}
        \item \textbf{self-note: see camera roll to fill this in with}
    \end{enumerate}
    Converting from \textbf{Cartesian Coordinates} to \textbf{Polar Coordinates}:
    \begin{enumerate}
        \item \textbf{self-note: see camera roll to fill this in with}
    \end{enumerate}
\end{definitionbox}

\section*{Additional Notes}
\begin{notebox}
Always check the domain of the parameter $t$ when solving problems involving parametric equations.
\end{notebox}

\section*{Further Visualization}
\begin{figure}[H]
    \centering
    \includegraphics[width=0.6\textwidth]{sample_image2.jpg}
    \caption{Additional visualization for parametric curves.}
    \label{fig:sample_image2}
\end{figure}

\end{document}
