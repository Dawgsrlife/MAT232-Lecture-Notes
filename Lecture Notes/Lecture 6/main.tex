\documentclass{article}
\usepackage{amsthm,amsmath,amsfonts,lipsum}
\usepackage[T1]{fontenc}
\usepackage{beramono}
\usepackage{listings}
\usepackage{fontawesome5}
\usepackage{adjustbox}
\usepackage{mathabx}
\usepackage{thmtools}
\usepackage{import}
\usepackage{graphicx}
\usepackage{setspace}
\usepackage{geometry}
\usepackage{physics}
\usepackage{float}
\usepackage[english]{babel}
\usepackage{framed}
\usepackage[dvipsnames,x11names]{xcolor}
\usepackage{tcolorbox}
\usepackage{fancyhdr}
\usepackage{hyperref}
\usepackage{booktabs}
\usepackage{enumitem}
\usepackage{cancel}
\usepackage{background}
\usepackage{units}
\usepackage{mathtools}
\usepackage{bm}
\usepackage{caption}
\usepackage{esvect}

% Configuring the background
\backgroundsetup{
  scale=1, % Optional, scale if needed
  color=black, % Optional, set the image color, can be omitted
  opacity=0.18, % Optional, adjust opacity for watermark effect
  angle=0,
  position=current page.center, % Center the image on the page
  contents={\includegraphics[width=1.75\paperwidth, height=1.75\paperheight, keepaspectratio]{ninym_ralei_leaf (watermarked by AlexanderTheMango)}} % Keeps aspect ratio and scales to fill the page
}

% Colours
\definecolor{darkgreen}{rgb}{0.0, 0.5, 0.0}
\definecolor{Firebrick}{rgb}{0.698, 0.132, 0.203}
\definecolor{Crimson}{rgb}{0.862745, 0.078431, 0.235294} % Crimson color
\definecolor{lightred}{rgb}{1.0, 0.819608, 0.819608} % Light red for background
\definecolor{MediumPurple}{rgb}{0.576, 0.439, 0.859}
\definecolor{chocolate}{rgb}{0.82, 0.41, 0.12} % Chocolate color definition
% Define the Navy color
\definecolor{Navy}{rgb}{0.0, 0.0, 0.5}

% Define custom tcolorbox styles for notes
\tcbuselibrary{skins, breakable}
\newtcolorbox{definitionbox}{colframe=RoyalBlue, colback=blue!5!white, title=Definition}
\newtcolorbox{examplebox}{colframe=ForestGreen, colback=green!5!white, title=Example}
\newtcolorbox{notebox}{colframe=RedOrange, colback=orange!5!white, title=Note}
\newtcolorbox{theorembox}{colframe=RoyalPurple, colback=purple!5!white, title=Theorem}

\newtcolorbox{propositionbox}{colframe=Goldenrod, colback=yellow!10!white, title=Proposition}
\newtcolorbox{remarkbox}{colframe=MidnightBlue, colback=blue!10!white, title=Remark}
\newtcolorbox{corollarybox}{colframe=OliveGreen, colback=green!10!white, title=Corollary}
\newtcolorbox{warningbox}{colframe=Crimson, colback=lightred, title=Warning}
\newtcolorbox{proofbox}{colframe=Black, colback=gray!10!white, title=Proof}
\newtcolorbox{questionbox}{colframe=Teal, colback=teal!10!white, title=Question}
\newtcolorbox{tipbox}{colframe=Goldenrod, colback=yellow!10!white, title=Tip}
\newtcolorbox{exercisebox}{colframe=darkgreen, colback=green!5!white, title=Exercise}
\newtcolorbox{solutionbox}{colframe=DodgerBlue4, colback=blue!5!white, title=Solution}
\newtcolorbox{algorithmbox}{colframe=Navy, colback=blue!10!white, title=Algorithm}
\newtcolorbox{conceptbox}{colframe=chocolate, colback=brown!10!white, title=Concept}
\newtcolorbox{illustrationbox}{colframe=Firebrick, colback=red!10!white, title=Illustration}
\newtcolorbox{intuitionbox}{colframe=MediumPurple, colback=purple!10!white, title=Intuition}
\newtcolorbox{answerbox}{colframe=RoyalBlue, colback=blue!10!white, title=Answer}

% Geometry settings
\geometry{letterpaper, portrait, includeheadfoot=true, hmargin=1in, vmargin=1in}
\onehalfspacing

% Header and footer
\pagestyle{fancy}
\fancyhf{}
\lhead{MAT232 - Lecture Notes}
\rhead{\thepage}
\lfoot{University of Toronto Mississauga}
\rfoot{\today}

% Document starts
\begin{document}
\renewcommand{\familydefault}{\rmdefault}

\begin{titlepage}
    \null % This is a TeX command that does nothing but is necessary for vfill to work correctly
    \vfill
    \begin{center}
        {\fontsize{40}{48}\selectfont \bfseries MAT232 - Lecture 3}
        \vspace{20pt} \\
        {\LARGE Polar Coordinates and the Arc Length of Parametric Curves} \\
        \vspace{20pt}
        \textbf{AlexanderTheMango}
        \vspace{8pt}
        \\ Prepared for January 13, 2025
    \end{center}
    \vfill
\end{titlepage}

\addcontentsline{toc}{section}{Title Page}

\setcounter{page}{0}
\newpage
\tableofcontents
\newpage

\phantomsection
\input{preliminary}
\addcontentsline{toc}{section}{Preliminary Concepts}

\begin{definitionbox}
    Given \( z = f(x, y) \) at a point \( (x_0, y_0) \) in the direction of the unit vector \( \vv{u} = \langle a, b \rangle \), the directional derivative of \( f \) at \( (x_0, y_0) \) in the direction of \( \vv{u} \) is
    \[ D_{\vv{u}} f(x_0, y_0) = \lim_{h \to 0} \frac{f(x_0 + ha, y_0 + hb) - f(x_0, y_0)}{h}. \]
\end{definitionbox}

\begin{notebox}
    Note that the directional derivative can also be written as
    \begin{align*}
        D_{\vv{u}} f(x, y) &= f_x(x, y) a + f_y(x, y) b \\
        &= \langle f_x(x, y), f_y(x, y) \rangle \cdot \langle a, b \rangle \\
        &= \nabla f(x, y) \cdot \vv{u}.
    \end{align*}
\end{notebox}

\begin{definitionbox}
    The gradient of \( f \) is the vector function \( \nabla f = \langle f_x, f_y \rangle \).

    \begin{conceptbox}
        The gradient of \( f \) is a vector that points in the direction of the greatest rate of increase of \( f \) at a point \( (x, y) \).

        \[ \nabla f(x, y) = \langle f_x(x, y), f_y(x, y) \rangle = \langle \pdv{f}{x}, \pdv{f}{y} \rangle = f_x(x, y) \hat{i} + f_y(x, y) \hat{j}. \]
    \end{conceptbox}
\end{definitionbox}

The max rate of increase of \( f \) at \( (x, y) \) is \( \norm{\nabla f(x, y)} \).

The direction of the max rate of increase of \( f \) at \( (x, y) \) is \( -\nabla f(x, y) \).

\begin{definitionbox}
    The max rate of decrease of \( f \) at \( (x, y) \) is \( -\norm{\nabla f(x, y)} \).
\end{definitionbox}

The min rate of increase of \( f \) at \( (x, y) \) is \( -\norm{\nabla f(x, y)} \).

The direction of the min rate of increase of \( f \) at \( (x, y) \) is \( \nabla f(x, y) \).

\begin{definitionbox}
    The min rate of decrease of \( f \) at \( (x, y) \) is \( \norm{\nabla f(x, y)} \).
\end{definitionbox}

\cleardoublepage
\phantomsection
\input{intolecturecontent}
\addcontentsline{toc}{section}{Lecture Content}

\normalsize

\setcounter{page}{1}

\section*{Term Test 2 Reminder!}
\addcontentsline{toc}{subsection}{Term Test 2 Reminder!}

\begin{notebox}
    Term Test 2 will be held on March 6, 2025 (Week 8), from 6:00 PM to 8:00 PM. The test will cover the following topics:
    \begin{itemize}
        \item Double integrals
        \item Domain of integration
        \item Clairut's theorem
        \item Greene's theorem
        \item Level curves
        \item Vector fields
        \item Line integrals
        \item Surface integrals
    \end{itemize}
\end{notebox}

\section*{Section 4.7: Max and Min Values}
\addcontentsline{toc}{subsection}{Section 4.7: Max and Min Values}

Recall from $1^{\text{st}}$ year calculus:
\begin{itemize}
    \item Local max/min: \( f'(c) = 0 \) or \( f'(c) \) does not exist.
    \item Absolute max/min: \( f(c) \geq f(x) \) or \( f(c) \leq f(x) \) for all \( x \) in the domain of \( f \).
    \item Critical point: \( f'(c) = 0 \) or \( f'(c) \) does not exist.
\end{itemize}

\begin{illustrationbox}
    \begin{figure}[H]
        \centering
        \includegraphics[width=0.5\textwidth]{sample_image.jpg}
        \caption{Max and Min Values}
    \end{figure}
\end{illustrationbox}

Given \( y = f(x), y' = f'(x) = 0 \) , find the critical numbers (cn):
\begin{itemize}
    \item Find \( f'(x) \).
    \item Solve \( f'(x) = 0 \) or \( f'(x) \) does not exist.
    \item The solutions are the critical numbers.
\end{itemize}

\begin{notebox}
    First derivative test:
    \begin{itemize}
        \item If \( f'(c) = 0 \) and \( f'(x) > 0 \) to the left of \( c \) and \( f'(x) < 0 \) to the right of \( c \), then \( f(c) \) is a local max.
        \item If \( f'(c) = 0 \) and \( f'(x) < 0 \) to the left of \( c \) and \( f'(x) > 0 \) to the right of \( c \), then \( f(c) \) is a local min.
        \item If \( f'(c) = 0 \) and \( f'(x) \) does not change sign, then the test is inconclusive.
    \end{itemize}
\end{notebox}

Given \( y' = f'(x), y'' = f''(x) = 0 \), find the critical  numbers (cn):
\begin{itemize}
    \item Find \( f'(x) \) and \( f''(x) \).
    \item Solve \( f'(x) = 0 \) or \( f'(x) \) does not exist.
    \item The solutions are the critical numbers.
\end{itemize}

\( f(x) \) is concave up if \( f''(x) > 0 \) and concave down if \( f''(x) < 0 \).

\begin{notebox}
    Second derivative test:
    \begin{itemize}
        \item If \( f'(c) = 0 \) and \( f''(c) > 0 \), then \( f(c) \) is a local min.
        \item If \( f'(c) = 0 \) and \( f''(c) < 0 \), then \( f(c) \) is a local max.
        \item If \( f'(c) = 0 \) and \( f''(c) = 0 \), then the test is inconclusive.
    \end{itemize}
\end{notebox}

Now in MAT232:
\begin{itemize}
    \item Local max/min: \( \nabla f = \langle 0, 0 \rangle \) or \( \nabla f \) does not exist.
    \item Absolute max/min: \( f(c) \geq f(x) \) or \( f(c) \leq f(x) \) for all \( x \) in the domain of \( f \).
    \item Critical point: \( \nabla f = \langle 0, 0 \rangle \) or \( \nabla f \) does not exist.
    \item Second derivative test: \( D > 0 \) and \( f_{xx} > 0 \) or \( D > 0 \) and \( f_{xx} < 0 \).
\end{itemize}

Suppose that the second partial derivatives of \( f \) are continuous in an open disk centered at \( (a, b) \). Moreover, suppose 

\end{document}
