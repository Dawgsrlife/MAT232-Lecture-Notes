\documentclass{article}
\usepackage{amsthm,amsmath,amsfonts,lipsum}
\usepackage[T1]{fontenc}
\usepackage{beramono}
\usepackage{listings}
\usepackage{fontawesome5}
\usepackage{adjustbox}
\usepackage{mathabx}
\usepackage{thmtools}
\usepackage{import}
\usepackage{graphicx}
\usepackage{setspace}
\usepackage{geometry}
\usepackage{physics}
\usepackage{float}
\usepackage[english]{babel}
\usepackage{framed}
\usepackage[dvipsnames,x11names]{xcolor}
\usepackage{tcolorbox}
\usepackage{fancyhdr}
\usepackage{hyperref}
\usepackage{booktabs}
\usepackage{enumitem}
\usepackage{cancel}
\usepackage{background}
\usepackage{units}

% Configuring the background
\backgroundsetup{
  scale=1, % Optional, scale if needed
  color=black, % Optional, set the image color, can be omitted
  opacity=0.18, % Optional, adjust opacity for watermark effect
  angle=0,
  position=current page.center, % Center the image on the page
  contents={\includegraphics[width=1.75\paperwidth, height=1.75\paperheight, keepaspectratio]{ninym_ralei_leaf (watermarked by AlexanderTheMango)}} % Keeps aspect ratio and scales to fill the page
}

% Colours
\definecolor{darkgreen}{rgb}{0.0, 0.5, 0.0}
\definecolor{Firebrick}{rgb}{0.698, 0.132, 0.203}
\definecolor{Crimson}{rgb}{0.862745, 0.078431, 0.235294} % Crimson color
\definecolor{lightred}{rgb}{1.0, 0.819608, 0.819608} % Light red for background
\definecolor{MediumPurple}{rgb}{0.576, 0.439, 0.859}
\definecolor{chocolate}{rgb}{0.82, 0.41, 0.12} % Chocolate color definition
% Define the Navy color
\definecolor{Navy}{rgb}{0.0, 0.0, 0.5}

% Define custom tcolorbox styles for notes
\tcbuselibrary{skins, breakable}
\newtcolorbox{definitionbox}{colframe=RoyalBlue, colback=blue!5!white, title=Definition}
\newtcolorbox{examplebox}{colframe=ForestGreen, colback=green!5!white, title=Example}
\newtcolorbox{notebox}{colframe=RedOrange, colback=orange!5!white, title=Note}
\newtcolorbox{theorembox}{colframe=RoyalPurple, colback=purple!5!white, title=Theorem}

\newtcolorbox{propositionbox}{colframe=Goldenrod, colback=yellow!10!white, title=Proposition}
\newtcolorbox{remarkbox}{colframe=MidnightBlue, colback=blue!10!white, title=Remark}
\newtcolorbox{corollarybox}{colframe=OliveGreen, colback=green!10!white, title=Corollary}
\newtcolorbox{warningbox}{colframe=Crimson, colback=lightred, title=Warning}
\newtcolorbox{proofbox}{colframe=Black, colback=gray!10!white, title=Proof}
\newtcolorbox{questionbox}{colframe=Teal, colback=teal!10!white, title=Question}
\newtcolorbox{tipbox}{colframe=Goldenrod, colback=yellow!10!white, title=Tip}
\newtcolorbox{exercisebox}{colframe=darkgreen, colback=green!5!white, title=Exercise}
\newtcolorbox{solutionbox}{colframe=DodgerBlue4, colback=blue!5!white, title=Solution}
\newtcolorbox{algorithmbox}{colframe=Navy, colback=blue!10!white, title=Algorithm}
\newtcolorbox{conceptbox}{colframe=chocolate, colback=brown!10!white, title=Concept}
\newtcolorbox{illustrationbox}{colframe=Firebrick, colback=red!10!white, title=Illustration}
\newtcolorbox{intuitionbox}{colframe=MediumPurple, colback=purple!10!white, title=Intuition}
\newtcolorbox{answerbox}{colframe=RoyalBlue, colback=blue!10!white, title=Answer}

% Geometry settings
\geometry{letterpaper, portrait, includeheadfoot=true, hmargin=1in, vmargin=1in}
\onehalfspacing

% Header and footer
\pagestyle{fancy}
\fancyhf{}
\lhead{MAT232 - Lecture Notes}
\rhead{\thepage}
\lfoot{University of Toronto Mississauga}
\rfoot{\today}

% Document starts
\begin{document}
\renewcommand{\familydefault}{\rmdefault}

\begin{titlepage}
    \null % This is a TeX command that does nothing but is necessary for vfill to work correctly
    \vfill
    \begin{center}
        {\fontsize{40}{48}\selectfont \bfseries MAT232 - Lecture 3}
        \vspace{20pt} \\
        {\LARGE Polar Coordinates and the Arc Length of Parametric Curves} \\
        \vspace{20pt}
        \textbf{AlexanderTheMango}
        \vspace{8pt}
        \\ Prepared for January 13, 2025
    \end{center}
    \vfill
\end{titlepage}


\setcounter{page}{0}
\newpage
\tableofcontents
\newpage

\input{preliminary}



\input{intolecturecontent}
\normalsize

\setcounter{page}{1}

\textbf{self-note: complete this according to the posted lecture notes}

\subsection*{SDT}
\begin{itemize}
    \item rel max
    \item rel min
    \item saddle point
    \item 
\end{itemize}

\section*{Section 14.8: Lagrange Multipliers}
\begin{illustrationbox}

\end{illustrationbox}
Absolute maximum? Find it using the gradient!

\[
    z = f(x, y)
\]

The gradient would be pointing upwards.

\[
  g(x, y) = k
\]
\( k \) is a scalar/constant \\
\noindent
\underline{Lagrange Multipliers} \\
\begin{examplebox}
    Given \( f(x, y) \) subject to a constant
    \[
        g(x, y) = k
    \]
    1: Solve
    (fix the stuff below later) \\
    % \[
    %     \begin{dcases}
    %         \nabla f(x, y) = \lambda \nabla g(x, y), &\text{ first }  ;\\
    %         g(x, y) = k, &\text{ second }  .
    %     \end{dcases}
    % \]
    2: \( \nabla g(x, y) \neq \overline{0} \); discard that \( (x, y) \). \\
    \textbf{self-note: finish this according to the posted lecture notes later}
    3: Plug all \( (x, y) \) 
\end{examplebox}

\begin{examplebox}
    Find the minimum fencing required if the area is \( 800 m^2 \).

    \begin{illustrationbox}
        \textbf{self-note: add the illustration of the fencing from the notes later on}
    \end{illustrationbox}

    \begin{solutionbox}
        Perimeter:
        \[
            P = 2y + x
        \]
        Area:
        \[
            800 = xy
        \]
        Note that the \( y \) is the constraint. \\
        \\
        So\dots
        \[
            y = \frac{800}{x}
        \]
        So\dots
        \begin{align*}
            P(x) &= 2  (\frac{800}{x}) + x \\
            &= 1600x^{-1} + x \\
            \implies P'(x) &= -1600x^{-2} + 1 \\
            \implies P''(x) &= 3200x^{-3} \\
            \implies P'(x) = 0 &= -\frac{1600}{x} + \frac{1}{1} \cdot \frac{x^2}{x^2} \\
            \implies 0 &= \frac{-1600 + x^2}{x^2}
        \end{align*}
        So\dots
        Num:
        \[
            0 = -1600 + x^2
        \]
        This means
        \[
            x = \pm 40
        \]
        so only
        \[
            x = 40.
        \]
        Note that denom
        \[
            x^2 = 0
        \]

        \begin{align*}
            P''(x) &= 3200x^{-3} \quad \dots \quad x = 40 \\
            \implies P''(40) &= \frac{3200}{40^3} > 0
        \end{align*}
        So, by SDT, rel min at \( x = 40 \).
        So\dots
        \begin{align*}
            800 &= xy \\
            \implies 800 &= 40y \\
            \implies 20 &= y
        \end{align*}
        Therefore, the minimum fencing needed is
        \[
            P = x + 2y = 40 + 2 \cdot 20 = 80 m.
        \]
    \end{solutionbox}
\end{examplebox}

Now, let's achieve the same result using Lagrange Multipliers!

\section*{Application of Lagrange Multipliers}
\addcontentsline{toc}{subsection}{Application of Lagrange Multiplers}

\begin{examplebox}
    Find the minimum fencing required if the area is \( 800 m^2 \).

    \begin{solutionbox}
        Perimeter:
        \[
            P = 2y + x \leftarrow f(x, y)
        \]
        Area:
        \[
            800 = xy \leftarrow \quad \text{constraint} \quad \leftarrow \quad g(x, y) = k
        \]
        \end{solutionbox}
    
        \subsubsection*{Step No. 1}
        Given \( P(x, y) = 2y + x \) subject to the constraint \( 800 = xy \).
    
        \begin{solutionbox}
        \begin{align*}
            \nabla P(x, y) &= \lambda \nabla g(x, y) \\
            \langle P_x, P_y \rangle &= \lambda \langle g_x, g_y \rangle \\
            \langle 1, 2 \rangle &= \lambda \langle y, x \rangle \\
            \textbf{self-note: fix this later}
            % systm of equations:
            % \begin{dcases}
            %     1 = \lambda y, &\text{ if \lambda \neq 0}  ;\\
            %     2 = \lambda x, &\text{ if }  .
            % \end{dcases}
            \implies \frac{1}{y} &= \lambda, \quad y \neq 0 \\
            \land \frac{2}{x} = \lambda
        \end{align*}
        Equate \( \lambda \):
        \[
            \frac{1}{y} = \frac{2}{x}
        \]
        \[
            x = 2y
        \]
        \subsubsection*{Constraint}
        \begin{align*}
            800 &= xy \\
            800 &= 2y \cdot y \\
            400 &= y^2 \\
            \pm 20 &= y \\
            \text{So\dots} \quad y &= 20 \\
            \text{So\dots} \quad x &= 2y \\
            x &= 2 \cdot 20 \\
            x &= 40
        \end{align*}
        So \( (40, 20) \).

        In this case, because \( x = \text{something}y \), we found the \( y \)-value.

        \textbf{self-note: don't forget to add step no.2 and step no. 3 from the posted lecture notes as well!}
    \end{solutionbox}
\end{examplebox}

\newpage  % maybe remove this newpage thing later

\begin{examplebox}
    Fidn the absolute max and absolute min of \( f (x, y) = 5x - 3y \) subject to the constraint \( x^2 + y^2 = 136 \).
    
    \begin{solutionbox}
        \[
            f(x, y) = 5x - 3y
        \]
        \[
            g(x, y) = x^2 + y^2 = 136 = k
        \]

        \underline{Step no. 1}
        \begin{align*}
            \nabla f(x, y) &= \lambda \nabla g(x, y) \\
            \langle 5, -3 \rangle &= \lambda \langle 2x, 2y \rangle \\
            \text{Notice that} \quad 5 = 2\lambda x &\land -3 = 2 \lambda x.
        \end{align*}
        \begin{itemize}
            \item \( \lambda \neq 0 \):
            \begin{itemize}
                \item \( \frac{5}{2x} = \lambda \quad \frac{-3}{2y} = \lambda \)
                \item \( 5y = -3x \) 
                \item \( y = \frac{-3x}{5} \)
            \end{itemize}
            Time for the constraint:
            \begin{align*}
                x^2 + y^2 &= 136 \\
                x^2 + \left( \frac{-3x}{5} \right)^2 &= 136 \\
                x^2 + \frac{9x^2}{25} &= 136 \\
                \frac{25x^2 + 9x^2}{25} &= 136 \\
                \frac{34x^2}{25} &= 136 \\
                34x^2 &= 136 \cdot 25 \\
                34x^2 &= 3400 \\
                x^2 &= 100 \\
                x &= \pm 10
            \end{align*}
        \end{itemize}
    \end{solutionbox}
\end{examplebox}

\end{document}
