\documentclass{article}
\usepackage{amsthm,amsmath,amsfonts,lipsum}
\usepackage[T1]{fontenc}
\usepackage{beramono}
\usepackage{listings}
\usepackage{fontawesome5}
\usepackage{adjustbox}
\usepackage{mathabx}
\usepackage{thmtools}
\usepackage{import}
\usepackage{graphicx}
\usepackage{setspace}
\usepackage{geometry}
\usepackage{physics}
\usepackage{float}
\usepackage[english]{babel}
\usepackage{framed}
\usepackage[dvipsnames,x11names]{xcolor}
\usepackage{tcolorbox}
\usepackage{fancyhdr}
\usepackage{hyperref}
\usepackage{booktabs}
\usepackage{enumitem}
\usepackage{cancel}
\usepackage{background}
\usepackage{units}

% Configuring the background
\backgroundsetup{
  scale=1, % Optional, scale if needed
  color=black, % Optional, set the image color, can be omitted
  opacity=0.18, % Optional, adjust opacity for watermark effect
  angle=0,
  position=current page.center, % Center the image on the page
  contents={\includegraphics[width=1.75\paperwidth, height=1.75\paperheight, keepaspectratio]{ninym_ralei_leaf (watermarked by AlexanderTheMango)}} % Keeps aspect ratio and scales to fill the page
}

% Colours
\definecolor{darkgreen}{rgb}{0.0, 0.5, 0.0}
\definecolor{Firebrick}{rgb}{0.698, 0.132, 0.203}
\definecolor{Crimson}{rgb}{0.862745, 0.078431, 0.235294} % Crimson color
\definecolor{lightred}{rgb}{1.0, 0.819608, 0.819608} % Light red for background
\definecolor{MediumPurple}{rgb}{0.576, 0.439, 0.859}
\definecolor{chocolate}{rgb}{0.82, 0.41, 0.12} % Chocolate color definition
% Define the Navy color
\definecolor{Navy}{rgb}{0.0, 0.0, 0.5}

% Define custom tcolorbox styles for notes
\tcbuselibrary{skins, breakable}
\newtcolorbox{definitionbox}{colframe=RoyalBlue, colback=blue!5!white, title=Definition}
\newtcolorbox{examplebox}{colframe=ForestGreen, colback=green!5!white, title=Example}
\newtcolorbox{notebox}{colframe=RedOrange, colback=orange!5!white, title=Note}
\newtcolorbox{theorembox}{colframe=RoyalPurple, colback=purple!5!white, title=Theorem}

\newtcolorbox{propositionbox}{colframe=Goldenrod, colback=yellow!10!white, title=Proposition}
\newtcolorbox{remarkbox}{colframe=MidnightBlue, colback=blue!10!white, title=Remark}
\newtcolorbox{corollarybox}{colframe=OliveGreen, colback=green!10!white, title=Corollary}
\newtcolorbox{warningbox}{colframe=Crimson, colback=lightred, title=Warning}
\newtcolorbox{proofbox}{colframe=Black, colback=gray!10!white, title=Proof}
\newtcolorbox{questionbox}{colframe=Teal, colback=teal!10!white, title=Question}
\newtcolorbox{tipbox}{colframe=Goldenrod, colback=yellow!10!white, title=Tip}
\newtcolorbox{exercisebox}{colframe=darkgreen, colback=green!5!white, title=Exercise}
\newtcolorbox{solutionbox}{colframe=DodgerBlue4, colback=blue!5!white, title=Solution}
\newtcolorbox{algorithmbox}{colframe=Navy, colback=blue!10!white, title=Algorithm}
\newtcolorbox{conceptbox}{colframe=chocolate, colback=brown!10!white, title=Concept}
\newtcolorbox{illustrationbox}{colframe=Firebrick, colback=red!10!white, title=Illustration}
\newtcolorbox{intuitionbox}{colframe=MediumPurple, colback=purple!10!white, title=Intuition}
\newtcolorbox{answerbox}{colframe=RoyalBlue, colback=blue!10!white, title=Answer}

% Geometry settings
\geometry{letterpaper, portrait, includeheadfoot=true, hmargin=1in, vmargin=1in}
\onehalfspacing

% Header and footer
\pagestyle{fancy}
\fancyhf{}
\lhead{MAT232 - Lecture Notes}
\rhead{\thepage}
\lfoot{University of Toronto Mississauga}
\rfoot{\today}

% Document starts
\begin{document}
\renewcommand{\familydefault}{\rmdefault}

\begin{titlepage}
    \null % This is a TeX command that does nothing but is necessary for vfill to work correctly
    \vfill
    \begin{center}
        {\fontsize{40}{48}\selectfont \bfseries MAT232 - Lecture 3}
        \vspace{20pt} \\
        {\LARGE Polar Coordinates and the Arc Length of Parametric Curves} \\
        \vspace{20pt}
        \textbf{AlexanderTheMango}
        \vspace{8pt}
        \\ Prepared for January 13, 2025
    \end{center}
    \vfill
\end{titlepage}


\setcounter{page}{0}
\newpage
\tableofcontents
\newpage

\input{preliminary}



\input{intolecturecontent}
\normalsize

\setcounter{page}{1}

\textbf{self-note: add the things from the beginning of the lecture here}

\newpage % remove this later

\begin{examplebox}
    Determine the point on the plane \( 4x - 2y + z = 1 \) that is the closest to the point \( (x_0, y_0, z_0) = (-2, -1, 5) \).

    \begin{solutionbox}
        \begin{align*}
            d &= \sqrt{(x - x_0)^2 + (y - y_0)^2 + (z - z_0)^2} \\
            d &= \sqrt{(x + 2)^2 + (y + 1)^2 + (z - 5)^2} \\
            d^2 &= (x + 2)^2 + (y + 1)^2 + (z - 5)^2 \\
        \end{align*}
        Consider that \( z = 1 - 4x + 2y \). Therefore, we have that
        \begin{align*}
            f_x &= 2(x + 2)'[1] + 2(-4x + 2y - 4)'[-4] \\
            &= 2(x + 2) - 8(-4x + 2y - 4) \\
            &= 2x + 4 + 32x - 16y + 32 \\
            &= 34x - 16y + 36 \\
            &= 0 \\
            f_y &= 2(y + 1)'[1] + 2(-4x + 2y - 4)'[2] \\
            &= 2(y + 1) + 4(-4x + 2y - 4) \\
            &= 2y + 2 - 16x + 8y - 16 \\
            &= -16x + 10y - 14 \\
            &= 0 \\
        \end{align*}
        Solve the system of equations using substitution. We have that
        \begin{align*}
            34x - 16y + 36 = 2(16x - 10y + 14) &= 0 \implies 8y = 17x + 18 \\
            -16x + 10y - 14 = 2(5y - 8x - 7) &= 0 \implies 5y - 8x - 7 = 0
        \end{align*}

        \textbf{finish this part according to the lecture notes lol}

    \end{solutionbox}
\end{examplebox}

\begin{solutionbox}
    Notice that the plane is given in the form \( Ax + By + Cz = D \), where \( A = 4 \), \( B = -2 \), \( C = 1 \), and \( D = 1 \). The normal vector to the plane is given by \( \vb{n} = \begin{pmatrix} 4 \\ -2 \\ 1 \end{pmatrix} \). The point on the plane that is closest to the point \( (x_0, y_0, z_0) = (-2, -1, 5) \) is the point on the plane that is closest to the point \( (x_0, y_0, z_0) = (-2, -1, 5) \). The vector from the point \( (x_0, y_0, z_0) = (-2, -1, 5) \) to the point on the plane is orthogonal to the normal vector of the plane. Therefore, the vector from the point \( (x_0, y_0, z_0) = (-2, -1, 5) \) to the point on the plane is given by \( \vb{v} = \begin{pmatrix} x - (-2) \\ y - (-1) \\ z - 5 \end{pmatrix} = \begin{pmatrix} x + 2 \\ y + 1 \\ z - 5 \end{pmatrix} \). Since \( \vb{v} \) is orthogonal to \( \vb{n} \), we have that \( \vb{v} \cdot \vb{n} = 0 \). Therefore, we have that
    \begin{align*}
        \begin{pmatrix} x + 2 \\ y + 1 \\ z - 5 \end{pmatrix} \cdot \begin{pmatrix} 4 \\ -2 \\ 1 \end{pmatrix} &= 0 \\
        4(x + 2) - 2(y + 1) + (z - 5) &= 0 \\
        4x + 8 - 2y - 2 + z - 5 &= 0 \\
        4x - 2y + z + 1 &= 0 \\
        4x - 2y + z &= -1
    \end{align*}
    Since the point on the plane that is closest to the point \( (x_0, y_0, z_0) = (-2, -1, 5) \) is the point on the plane that is closest to the point \( (x_0, y_0, z_0) = (-2, -1, 5) \), we have that the point on the plane that is closest to the point \( (x_0, y_0, z_0) = (-2, -1, 5) \) is \( (-2, -1, 5) \).
\end{solutionbox}

In first year calculus, we learned that the derivative of a function gives us the slope of the tangent line to the function at a point. In this course, we will learn that the gradient of a function gives us the direction of the steepest ascent of the function at a point. The gradient of a function is a vector that points in the direction of the steepest ascent of the function at a point. The gradient of a function is given by the vector of partial derivatives of the function. The gradient of a function is denoted by \( \nabla f \) or \( \grad f \). The gradient of a function is given by
\begin{align*}
    \nabla f(x, y, z) &= \left \langle \pdv{f}{x}, \pdv{f}{y}, \pdv{f}{z} \right \rangle
\end{align*}

\section*{Revisiting the Extreme Value Theorem}
\addcontentsline{toc}{subsection}{Revisiting the Extreme Value Theorem}

Recall that in $1^{\text{st}}$ year calculus, the extreme value theorem is defined as follows: \\
If $f$ is continuous on a closed interval $[a, b]$, then $f$ attains both a maximum and a minimum value on $[a, b]$. \\
\\
Now, in MAT232, we will extend this theorem to functions of two variables.

\begin{theorembox}
    If $f$ is continuous on a closed and bounded region $D$ in $\mathbb{R}^2$, then $f$ attains both an absolute maximum and an absolute minimum value at some points \( (x_1, y_1) \) and \( (x_2, y_2) \) in the region $D$.
\end{theorembox}

\begin{examplebox}
    Find the absolute maximum and minimum values of the function \( f(x, y) = 3xy - 6x - 3y + 7 \) on the closed triangular region R with vertices at \( (0, 0) \), \( (3, 0) \), and \( (0, 5) \).

    \textbf{self-note: be sure to also add the illustration for this example}

    \begin{solutionbox}
        \subsubsection*{Step 1: Find the Critical Points of the Function}
        To find the critical points of the function, we need to find the points where the gradient of the function is equal to the zero vector. The gradient of the function is given by
        \begin{align*}
            \nabla f(x, y) &= \left \langle \pdv{f}{x}, \pdv{f}{y} \right \rangle \\
            &= \left \langle 3y - 6, 3x - 3 \right \rangle
        \end{align*}
        Setting the gradient of the function equal to the zero vector, we have that
        \begin{align*}
            3y - 6 &= 0 \\
            3x - 3 &= 0
        \end{align*}
        Solving the system of equations, we have that
        \begin{align*}
            3y - 6 &= 0 \implies 3y = 6 \implies y = 2 \\
            3x - 3 &= 0 \implies 3x = 3 \implies x = 1
        \end{align*}
        Therefore, the critical point of the function is \( (1, 2) \). \\
        \\
        \textit{...cont'd...}
    \end{solutionbox}
\end{examplebox}
\begin{examplebox}
    \begin{solutionbox}
        \textit{...cont'd...}
        \subsubsection*{Step 2: Find the Maximum and Minimum Values of the Function on the Boundary of the Region}
        \( L_1: x = 0, \quad 0 \leq y \leq 5 \)
        \begin{align*}
            f(x, y) &= 3xy - 6x - 3y + 7 \\
            f(0, y) &= 3(0)y - 6(0) - 3y + 7 \\
            &= -3y + 7
            f_y = -3 \neq 0 \implies \text{no critical points}
        \end{align*}
        \( L_2: y = 0, \quad 0 \leq x \leq 3 \)
        \begin{align*}
            f(x, y) &= 3xy - 6x - 3y + 7 \\
            f(x, 0) &= 3x(0) - 6x - 3(0) + 7 \\
            &= -6x + 7
            f_x = -6 \neq 0 \implies \text{no critical points}
        \end{align*}
        \( L_3: \text{slope} = \dfrac{\text{rise}}{\text{run}} = \dfrac{\Delta 5}{\Delta 3} = \dfrac{5 - 0}{0 - 3} = -\dfrac{5}{3} \)
        So,
        \begin{align*}
            y &= mx + b, \quad \text{so } b = 5 \\
            y &= -\dfrac{5}{3}x + 5, \quad 0 \leq x \leq 3
        \end{align*}
        \begin{align*}
            f(x, y) &= 3xy - 6x - 3y + 7 \\
            f(x, -\dfrac{5}{3}x + 5) &= 3x(-\dfrac{5}{3}x + 5) - 6x - 3(-\dfrac{5}{3}x + 5) + 7 \\
            &= -5x^2 + 15x - 6x + 5x - 15 + 7 \\
            &= -5x^2 + 14x - 8
        \end{align*}
        \begin{align*}
            f_x &= -10x + 14 = 0 \\
            -10x &= -14 \\
            x &= \dfrac{7}{5} \\
        \end{align*}
        Since \( \dfrac{7}{5} \) is in the interval \( 0 \leq x \leq 3 \), we have that \( x = \dfrac{7}{5} \). \\
        \\
        So\dots
        \begin{align*}
            y &= -\dfrac{5}{3}(\dfrac{7}{5}) + 5 \\
            &= -\dfrac{7}{3} + 5 \\
            &= \dfrac{8}{3}
        \end{align*}
        Therefore, the critical point of the function is \( (\dfrac{7}{5}, \dfrac{8}{3}) \).
    \end{solutionbox}
    \textit{...cont'd...}
\end{examplebox}
\begin{examplebox}
    \textit{...cont'd...}
    \begin{solutionbox}
        \subsubsection*{Step 3: Find the Maximum and Minimum Values of the Function at the Critical Points and on the Boundary of the Region}
        The points where the function attains its maximum and minimum values are the critical points of the function and the points on the boundary of the region. These points are \( (1, 2) \), \( (\frac{7}{5}, \frac{8}{3}) \), \( (0, 0) \), \( (0, 5) \), and \( (3, 0) \). We need to evaluate the function at these points to determine the maximum and minimum values of the function. \\
        \begin{align*}
            f(1, 2) &= 3(1)(2) - 6(1) - 3(2) + 7 \\
            &= 6 - 6 - 6 + 7 \\
            &= 1
        \end{align*}
        \begin{align*}
            f \left( \dfrac{7}{5}, \dfrac{8}{3} \right) &= 3 \left( \dfrac{7}{5} \right) \left( \dfrac{8}{3} \right) - 6 \left( \dfrac{7}{5} \right) - 3 \left( \dfrac{8}{3} \right) + 7 \\
            &= \dfrac{\cancelto{3}{21}}{5} \cdot \dfrac{8}{\cancelto{1}{3}} - \dfrac{42}{5} - 8 + 7 \\
            &= \dfrac{56}{5} - \dfrac{42}{5} - 8 + 7 \\
            &= \dfrac{14}{5} - 8 + 7 \\
            &= \dfrac{14}{5} - \dfrac{40}{5} + \dfrac{35}{5} \\
            &= -\dfrac{26}{5} + \dfrac{35}{5} \\
            &= \dfrac{9}{5}
        \end{align*}
        \begin{align*}
            f(0, 0) &= 3(0)(0) - 6(0) - 3(0) + 7 \\
            &= 0 - 0 - 0 + 7 \\
            &= 7
        \end{align*}
        \begin{align*}
            f(0, 5) &= 3(0)(5) - 6(0) - 3(5) + 7 \\
            &= 0 - 0 - 15 + 7 \\
            &= -8
        \end{align*}
        \begin{align*}
            f(3, 0) &= 3(3)(0) - 6(3) - 3(0) + 7 \\
            &= 0 - 18 - 0 + 7 \\
            &= -11
        \end{align*}
        Therefore, the absolute maximum value of the function is \( 7 \) and the absolute minimum value of the function is \( -11 \).
    \end{solutionbox}
\end{examplebox}

\end{document}
