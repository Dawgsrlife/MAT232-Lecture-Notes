\documentclass{article}
\usepackage{amsthm,amsmath,amsfonts,lipsum,amssymb}
\usepackage[T1]{fontenc}
\usepackage{beramono}
\usepackage{listings}
\usepackage{fontawesome5}
\usepackage{adjustbox}
\usepackage{mathabx}
\usepackage{thmtools}
\usepackage{import}
\usepackage{graphicx}
\usepackage{setspace}
\usepackage{geometry}
\usepackage{physics}
\usepackage{float}
\usepackage[english]{babel}
\usepackage{framed}
\usepackage[dvipsnames,x11names]{xcolor}
\usepackage{tcolorbox}
\usepackage{fancyhdr}
\usepackage{hyperref}
\usepackage{booktabs}
\usepackage{enumitem}
\usepackage{cancel}
\usepackage{background}
\usepackage{units}

% Configuring the background
\backgroundsetup{
  scale=1, % Optional, scale if needed
  color=black, % Optional, set the image color, can be omitted
  opacity=0.18, % Optional, adjust opacity for watermark effect
  angle=0,
  position=current page.center, % Center the image on the page
  contents={\includegraphics[width=1.75\paperwidth, height=1.75\paperheight, keepaspectratio]{ninym_ralei_leaf (watermarked by AlexanderTheMango)}} % Keeps aspect ratio and scales to fill the page
}

% Colours
\definecolor{darkgreen}{rgb}{0.0, 0.5, 0.0}
\definecolor{Firebrick}{rgb}{0.698, 0.132, 0.203}
\definecolor{Crimson}{rgb}{0.862745, 0.078431, 0.235294} % Crimson color
\definecolor{lightred}{rgb}{1.0, 0.819608, 0.819608} % Light red for background
\definecolor{MediumPurple}{rgb}{0.576, 0.439, 0.859}
\definecolor{chocolate}{rgb}{0.82, 0.41, 0.12} % Chocolate color definition
% Define the Navy color
\definecolor{Navy}{rgb}{0.0, 0.0, 0.5}

% Define custom tcolorbox styles for notes
\tcbuselibrary{skins, breakable}
\newtcolorbox{definitionbox}{colframe=RoyalBlue, colback=blue!5!white, title=Definition}
\newtcolorbox{examplebox}{colframe=ForestGreen, colback=green!5!white, title=Example}
\newtcolorbox{notebox}{colframe=RedOrange, colback=orange!5!white, title=Note}
\newtcolorbox{theorembox}{colframe=RoyalPurple, colback=purple!5!white, title=Theorem}

\newtcolorbox{propositionbox}{colframe=Goldenrod, colback=yellow!10!white, title=Proposition}
\newtcolorbox{remarkbox}{colframe=MidnightBlue, colback=blue!10!white, title=Remark}
\newtcolorbox{corollarybox}{colframe=OliveGreen, colback=green!10!white, title=Corollary}
\newtcolorbox{warningbox}{colframe=Crimson, colback=lightred, title=Warning}
\newtcolorbox{proofbox}{colframe=Black, colback=gray!10!white, title=Proof}
\newtcolorbox{questionbox}{colframe=Teal, colback=teal!10!white, title=Question}
\newtcolorbox{tipbox}{colframe=Goldenrod, colback=yellow!10!white, title=Tip}
\newtcolorbox{exercisebox}{colframe=darkgreen, colback=green!5!white, title=Exercise}
\newtcolorbox{solutionbox}{colframe=DodgerBlue4, colback=blue!5!white, title=Solution}
\newtcolorbox{algorithmbox}{colframe=Navy, colback=blue!10!white, title=Algorithm}
\newtcolorbox{conceptbox}{colframe=chocolate, colback=brown!10!white, title=Concept}
\newtcolorbox{illustrationbox}{colframe=Firebrick, colback=red!10!white, title=Illustration}
\newtcolorbox{intuitionbox}{colframe=MediumPurple, colback=purple!10!white, title=Intuition}
\newtcolorbox{answerbox}{colframe=RoyalBlue, colback=blue!10!white, title=Answer}

% Geometry settings
\geometry{letterpaper, portrait, includeheadfoot=true, hmargin=1in, vmargin=1in}
\onehalfspacing

% Header and footer
\pagestyle{fancy}
\fancyhf{}
\lhead{MAT232 - Lecture Notes}
\rhead{\thepage}
\lfoot{University of Toronto Mississauga}
\rfoot{\today}

% Document starts
\begin{document}
\renewcommand{\familydefault}{\rmdefault}

\begin{titlepage}
    \null % This is a TeX command that does nothing but is necessary for vfill to work correctly
    \vfill
    \begin{center}
        {\fontsize{40}{48}\selectfont \bfseries MAT232 - Lecture 3}
        \vspace{20pt} \\
        {\LARGE Polar Coordinates and the Arc Length of Parametric Curves} \\
        \vspace{20pt}
        \textbf{AlexanderTheMango}
        \vspace{8pt}
        \\ Prepared for January 13, 2025
    \end{center}
    \vfill
\end{titlepage}


\setcounter{page}{0}
\newpage
\tableofcontents
\newpage

\input{preliminary}



\input{intolecturecontent}
\normalsize

\setcounter{page}{1}

\begin{notebox}
Remember that term test 1 is on Thursday, January 30$^{\text{th}}$, 2025 — from 6-8pm! \\
\\
Good luck studying!
\end{notebox}

\section*{Section 2.5: Lines and Planes}
\addcontentsline{toc}{section}{Section 2.5: Lines and Planes}
\subsection*{Recall from high school...}
\addcontentsline{toc}{subsection}{Section 2.5: Lines and Planes}
The line equation is defined by
\[
    y = mx + b
\]
where $m$ is the slope and $b$ is the $y$-intercept. The slope is defined as the change in $y$ over the change in $x$. The $y$-intercept is the point where the line crosses the $y$-axis. \\
\\
Alternatively, there was point-slope form, which is defined as
\[
    y - y_1 = m(x - x_1)
\]
where $(x_1, y_1)$ is a point on the line. \\

\subsection*{Now, in MAT232, exploring the 3D world...}
\addcontentsline{toc}{subsection}{Now, in MAT232, exploring the 3D world...}
\begin{definitionbox}
In 3D, we have a line equation defined by
\[
    \begin{aligned}
        x &= x_0 + at \\
        y &= y_0 + bt \\
        z &= z_0 + ct
    \end{aligned}
\]
where $(x_0, y_0, z_0)$ is a point on the line and $(a, b, c)$ is the direction vector. The parameter $t$ is a scalar.
\end{definitionbox}

\begin{definitionbox}
Vector equation:
\[
    \overline{r} = \overline{r_0} + t\overline{v}
\]
\[
    <x, y, z> = <x_0, y_0, z_0> + t<v_1, v_2, v_3>
\]
\end{definitionbox}
\begin{definitionbox}
This is also written as:
\[
    x = x_0 + at
\]
\[
    y = y_0 + bt
\]
\[
    z = z_0 + ct
\]
\end{definitionbox}

\begin{examplebox}
What does \( <x, y, z> = <-1, 0, 2> + t^2 <2, 10, -8> \) represent? Note that \( t \in \mathbb{R} \) is a scalar.
\begin{solutionbox}
It represents the line in 3D space that passes through the point \( (-1, 0, 2) \) and has direction vector \( <2, 10, -8> \) (or is parallel to this direction vector).
\end{solutionbox}
\end{examplebox}

Try this at home: what about this one?
\begin{exercisebox}
What does \( <x, y, z> = <-1, 0, 2> + 2t^3 <1, 5, -4> \) represent?
\end{exercisebox}

\begin{examplebox}
(A): Find the parametric equations of the line \( L \) that pass through the points \( A(2, 4 -1) \) and \( B(5, 0, 7) \).
(B): Does this line intersect the \( xy \)-plane? If so, where? (Hint: \( z = 0 \).)
\begin{solutionbox}
(A): The direction vector is \( <5 - 2, 0 - 4, 7 - (-1)> = <3, -4, 8> \). The parametric equations are
\[
    \begin{aligned}
        x &= 2 + 3t \\
        y &= 4 - 4t \\
        z &= -1 + 8t
    \end{aligned}
\]
Vector equation:
\[
    \overline{r} = \overline{r_0} + t\overline{v}
\]
\[
    <x, y, z> = <2, 4, -1> + t<3, -4, 8>
\]
(B): To find the intersection with the \( xy \)-plane, we set \( z = 0 \) and solve for \( t \):
\[
    -1 + 8t = 0 \implies t = \frac{1}{8}
\]
Substitute \( t = \frac{1}{8} \) into the parametric equations to find the intersection point:
\[
    \begin{aligned}
        x &= 2 + 3\left(\frac{1}{8}\right) = \frac{19}{8} \\
        y &= 4 - 4\left(\frac{1}{8}\right) = \frac{7}{2} \\
        z &= -1 + 8\left(\frac{1}{8}\right) = 0
    \end{aligned}
\]
Thus, the line intersects the \( xy \)-plane at the point \( \left(\frac{19}{8}, \frac{7}{2}, 0\right) \).
\begin{answerbox}
The parametric equations of the line are
\[
    \begin{aligned}
        x &= 2 + 3t \\
        y &= 4 - 4t \\
        z &= -1 + 8t
    \end{aligned}
\]
The line intersects the \( xy \)-plane at the point \( \left(\frac{19}{8}, \frac{7}{2}, 0\right) \).
\end{answerbox}
\end{solutionbox}
\end{examplebox}

\subsection*{2 Lines in 3D}
\addcontentsline{toc}{subsection}{2 Lines in 3D}
\begin{remarkbox}
Two lines in 3D are either parallel, intersecting, or skew. Skew lines are lines that are not parallel and do not intersect.
\begin{itemize}
    \item Parallel lines have the same direction vector.
    \item Intersecting lines have the same direction vector and a point in common.
    \item Skew lines have different direction vectors.
\end{itemize}
\end{remarkbox}

\begin{notebox}
\begin{enumerate}
    \item Can be parallel?
    \item intersect at a point?
    \item Skewed?
\end{enumerate}
\end{notebox}

\begin{tipbox}
To determine if two lines are parallel, intersecting, or skew, we can compare the direction vectors and points on the lines.
\end{tipbox}

Let's try an example:
\begin{examplebox}
Let \( L_1 \) and \( L_2 \) be the lines defined as:
\[
    \begin{aligned}
        L_1: x &= 1 + 4t \\
        y &= 5 - 4t \\
        z &= -1 + 6t \\
        L_2: <x, y, z> &= <2, 4, 5> + s<8, -3, 1>
    \end{aligned}
\]
(A): Are the lines parallel, intersecting, or skew? \\
(B): If they intersect, find the point of intersection.

\begin{solutionbox}
(A): The direction vector of \( L_1 \) is \( <4, -4, 6> \) and the direction vector of \( L_2 \) is \( <8, -3, 1> \). Since the direction vectors are not the same, the lines are skew. \\
Specifically,
\[
    \overline{v_1} \stackrel{?}{=} k\overline{v_2}
\]
\[
    \langle 4, -4, 6 \rangle \stackrel{?}{\neq} k \langle 8, -3, 1 \rangle
\]
No, \( L_1 \) is not \( \parallel \) to \( L_2 \). \\

\begin{notebox}
There are two ways to check if two lines are parallel:
\begin{enumerate}
    \item \( \overline{v_1} = k\overline{v_2} \), \( k \) is a scalar.
    \item \( \overline{v_1} \times \overline{v_2} = \overline{0} \).
\end{enumerate}
\end{notebox}

(B):
Notice that
\[
    L_1: x = 1 + 4t, y = 5 - 4t, z = -1 + 6t
\]
\[
    L_2: x = 2 + 8s, y = 4 - 3s, z = 5 + s
\]
Equate the \( x \), \( y \), and \( z \) components of the two lines to find the point of intersection:
\[
    \begin{aligned}
        1 + 4t &= 2 + 8s \quad \textcircled{1} \\
        5 - 4t &= 4 - 3s \quad \textcircled{2} \\
        -1 + 6t &= 5 + s \quad \textcircled{3}
    \end{aligned}
\]
So,
\[
    6 = 6 + 5s \implies s = 0
\]
Using \( \textcircled{1} \),
\[
    1 + 4t = 2 \implies t = \frac{1}{4}.
\]
Now, check \( s = 0 \) and \( t = \frac{1}{4} \) in \textcircled{3} (\( LHS = RHS \)):
\[
    -1 + 5\left(\frac{1}{4}\right) = 5 + 0
\]
Clearly,
\[
    \frac{1}{4} \neq 5.
\]
Therefore, the lines do not intersect. \\
The lines are skewed.
\begin{answerbox}
The lines are skew and do not intersect.
\end{answerbox}

\end{solutionbox}

\end{examplebox}

\section*{Planes}
\addcontentsline{toc}{section}{Planes}
\begin{definitionbox}
In 3D, we can define a plane using the equation
\[
    Ax + By + Cz = D
\]
or alternative, we can define the plane using the parametric form
\[
    \begin{aligned}
        x &= x_0 + su + tv \\
        y &= y_0 + su + tv \\
        z &= z_0 + su + tv
    \end{aligned}
\]
where \( (x_0, y_0, z_0) \) is a point on the plane and \( (u, v) \) are the direction vectors. The parameter \( s \) and \( t \) are scalars.
\end{definitionbox}

Check out the \( x \) plane:
\[
    x = a \text{ fixed}
\]
Check out the \( y \) plane:
\[
    y = b \text{ fixed}
\]
Check out the \( z \) plane:
\[
    z = c \text{ fixed}
\]

\subsection*{Idea}
\addcontentsline{toc}{subsection}{Idea}
\begin{illustrationbox}
\textbf{self-note: add the illustration here from the camera roll}
\end{illustrationbox}
\begin{notebox}
\begin{enumerate}
    \item Point on plane - \( P(x_0, y_0, z_0) \)
    \item Vector living on the plane:
    \[
        \overline{r} - \overline{r_0}
    \]
    \item Normal vector (90 degrees to the plane): \( \overline{n} = \langle A, B, C \rangle \)
\end{enumerate}
\end{notebox}

\begin{definitionbox}
The equation of the plane is defined by
\[
    \overline{n} \cdot (\overline{r} - \overline{r_0}) = 0.
\]
\[
    \langle A, B, C \rangle \cdot \langle x, y, z \rangle - \langle <x_0, y_0, z_0 \rangle = 0
\]
\[
    \langle A, B, C \rangle \cdot \langle x - x_0, y - y_0, z - z_0 \rangle = 0
\]
\[
    A(x - x_0) + B(y - y_0) + C(z - z_0) = 0
\]
This is the scalar equation of the plane.
\end{definitionbox}

In vector form, we can define the line as
\[
    \begin{bmatrix}
        x \\ y \\ z
    \end{bmatrix} = \begin{bmatrix}
        x_0 \\ y_0 \\ z_0
    \end{bmatrix} + t \begin{bmatrix}
        a \\ b \\ c
    \end{bmatrix}
\]
where $(x_0, y_0, z_0)$ is a point on the line and $(a, b, c)$ is the direction vector. The parameter $t$ is a scalar. \\
\\
The vector equation can also be written as
\[
    \begin{bmatrix}
        x \\ y \\ z
    \end{bmatrix} = \begin{bmatrix}
        x_0 \\ y_0 \\ z_0
    \end{bmatrix} + t \vec{d}
\]
where $\vec{d} = \begin{bmatrix}
    a \\ b \\ c
\end{bmatrix}$ is the direction vector. \\
\\
We can also define the line using the symmetric form
\[
    \frac{x - x_0}{a} = \frac{y - y_0}{b} = \frac{z - z_0}{c}
\]
where $(x_0, y_0, z_0)$ is a point on the line and $(a, b, c)$ is the direction vector. \\
\\
In 3D, we can also define a plane using the equation
\[
    Ax + By + Cz = D
\]
where $(A, B, C)$ is the normal vector to the plane. The normal vector is perpendicular to the plane. The point $(x, y, z)$ is a point on the plane. The scalar $D$ is the distance from the origin to the plane. \\
\\
Alternatively, we can define the plane using the parametric form
\[
    \begin{aligned}
        x &= x_0 + su + tv \\
        y &= y_0 + su + tv \\
        z &= z_0 + su + tv
    \end{aligned}
\]
where $(x_0, y_0, z_0)$ is a point on the plane and $(u, v)$ are the direction vectors. The parameter $s$ and $t$ are scalars. \\
\\

\end{document}
