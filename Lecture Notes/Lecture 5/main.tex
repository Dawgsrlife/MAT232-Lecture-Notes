\documentclass{article}
\usepackage{amsthm,amsmath,amsfonts,lipsum}
\usepackage[T1]{fontenc}
\usepackage{beramono}
\usepackage{listings}
\usepackage{fontawesome5}
\usepackage{adjustbox}
\usepackage{mathabx}
\usepackage{thmtools}
\usepackage{import}
\usepackage{graphicx}
\usepackage{setspace}
\usepackage{geometry}
\usepackage{physics}
\usepackage{float}
\usepackage[english]{babel}
\usepackage{framed}
\usepackage[dvipsnames,x11names]{xcolor}
\usepackage{tcolorbox}
\usepackage{fancyhdr}
\usepackage{hyperref}
\usepackage{booktabs}
\usepackage{enumitem}
\usepackage{cancel}
\usepackage{background}
\usepackage{units}

% Configuring the background
\backgroundsetup{
  scale=1, % Optional, scale if needed
  color=black, % Optional, set the image color, can be omitted
  opacity=0.18, % Optional, adjust opacity for watermark effect
  angle=0,
  position=current page.center, % Center the image on the page
  contents={\includegraphics[width=1.75\paperwidth, height=1.75\paperheight, keepaspectratio]{ninym_ralei_leaf (watermarked by AlexanderTheMango)}} % Keeps aspect ratio and scales to fill the page
}

% Colours
\definecolor{darkgreen}{rgb}{0.0, 0.5, 0.0}
\definecolor{Firebrick}{rgb}{0.698, 0.132, 0.203}
\definecolor{Crimson}{rgb}{0.862745, 0.078431, 0.235294} % Crimson color
\definecolor{lightred}{rgb}{1.0, 0.819608, 0.819608} % Light red for background
\definecolor{MediumPurple}{rgb}{0.576, 0.439, 0.859}
\definecolor{chocolate}{rgb}{0.82, 0.41, 0.12} % Chocolate color definition
% Define the Navy color
\definecolor{Navy}{rgb}{0.0, 0.0, 0.5}

% Define custom tcolorbox styles for notes
\tcbuselibrary{skins, breakable}
\newtcolorbox{definitionbox}{colframe=RoyalBlue, colback=blue!5!white, title=Definition}
\newtcolorbox{examplebox}{colframe=ForestGreen, colback=green!5!white, title=Example}
\newtcolorbox{notebox}{colframe=RedOrange, colback=orange!5!white, title=Note}
\newtcolorbox{theorembox}{colframe=RoyalPurple, colback=purple!5!white, title=Theorem}

\newtcolorbox{propositionbox}{colframe=Goldenrod, colback=yellow!10!white, title=Proposition}
\newtcolorbox{remarkbox}{colframe=MidnightBlue, colback=blue!10!white, title=Remark}
\newtcolorbox{corollarybox}{colframe=OliveGreen, colback=green!10!white, title=Corollary}
\newtcolorbox{warningbox}{colframe=Crimson, colback=lightred, title=Warning}
\newtcolorbox{proofbox}{colframe=Black, colback=gray!10!white, title=Proof}
\newtcolorbox{questionbox}{colframe=Teal, colback=teal!10!white, title=Question}
\newtcolorbox{tipbox}{colframe=Goldenrod, colback=yellow!10!white, title=Tip}
\newtcolorbox{exercisebox}{colframe=darkgreen, colback=green!5!white, title=Exercise}
\newtcolorbox{solutionbox}{colframe=DodgerBlue4, colback=blue!5!white, title=Solution}
\newtcolorbox{algorithmbox}{colframe=Navy, colback=blue!10!white, title=Algorithm}
\newtcolorbox{conceptbox}{colframe=chocolate, colback=brown!10!white, title=Concept}
\newtcolorbox{illustrationbox}{colframe=Firebrick, colback=red!10!white, title=Illustration}
\newtcolorbox{intuitionbox}{colframe=MediumPurple, colback=purple!10!white, title=Intuition}
\newtcolorbox{answerbox}{colframe=RoyalBlue, colback=blue!10!white, title=Answer}

% Define the blank box
\newtcolorbox{blankbox}{
  colframe=white,        % Frame color is white (invisible)
  colback=white!5,      % 25% opacity white background
  boxrule=0pt,           % No border
  title=,                % No title
  rounded corners        % Rounded corners
}

% Geometry settings
\geometry{letterpaper, portrait, includeheadfoot=true, hmargin=1in, vmargin=1in}
\onehalfspacing

% Header and footer
\pagestyle{fancy}
\fancyhf{}
\lhead{MAT232 - Lecture Notes}
\rhead{\thepage}
\lfoot{University of Toronto Mississauga}
\rfoot{\today}

% Document starts
\begin{document}
\renewcommand{\familydefault}{\rmdefault}

\begin{titlepage}
    \null % This is a TeX command that does nothing but is necessary for vfill to work correctly
    \vfill
    \begin{center}
        {\fontsize{40}{48}\selectfont \bfseries MAT232 - Lecture 3}
        \vspace{20pt} \\
        {\LARGE Polar Coordinates and the Arc Length of Parametric Curves} \\
        \vspace{20pt}
        \textbf{AlexanderTheMango}
        \vspace{8pt}
        \\ Prepared for January 13, 2025
    \end{center}
    \vfill
\end{titlepage}

\addcontentsline{toc}{section}{Title Page}

\setcounter{page}{0}
\newpage
\tableofcontents
\newpage

\phantomsection
\input{preliminary}
\addcontentsline{toc}{section}{Preliminary Concepts}

\section*{Conic Sections}
\addcontentsline{toc}{subsection}{Conic Sections}

\begin{conceptbox}
\textbf{Definition of Conic Sections:}  
Conic sections are the curves formed by the intersection of a plane with a double-napped cone. The type of curve depends on the angle of the plane relative to the cone:
\begin{itemize}
    \item \textit{Circle:} The plane is perpendicular to the cone's axis.
    \item \textit{Ellipse:} The plane intersects one nappe of the cone but is not perpendicular to the axis.
    \item \textit{Parabola:} The plane is parallel to a generator of the cone.
    \item \textit{Hyperbola:} The plane intersects both nappes of the cone.
\end{itemize}
\begin{figure}[H]
    \centering
    \includegraphics[width=0.8\textwidth]{double-napped cone to 2D shapes.png}
    \caption{Conic sections formed by the intersection of a plane with a double-napped cone.}
\end{figure}
\end{conceptbox}

\section*{Ellipse}
\addcontentsline{toc}{subsection}{Ellipse}
\begin{definitionbox}
An \textbf{ellipse} is the set of all points in a plane such that the sum of their distances to two fixed points (called the \textit{foci}) is constant.
\begin{figure}[H]
    \centering
    \includegraphics[width=0.65\textwidth]{ellipse.jpg}
    \caption{Diagram of an ellipse.}
\end{figure}

\begin{intuitionbox} 
    Imagine looping a circular string around two fixed points \( F_1 \) and \( F_2 \) on a plane and pulling it taut (fully stretched without slack) with a pencil. As you move the pencil while keeping the string tight, the traced shape forms an ellipse. This method is commonly used for drawing ellipses with nails and string.
    \begin{figure}[H]
        \centering
        \includegraphics[width=0.4\textwidth]{ellipse_string_example.png}
        \caption{Drawing an ellipse with nails and string.}
    \end{figure}
\end{intuitionbox}
\end{definitionbox}

\subsection*{Standard Forms of an Ellipse}
\addcontentsline{toc}{subsubsection}{Standard Forms of an Ellipse}

\begin{definitionbox}
The equation of an ellipse depends on the orientation of its major axis:

\begin{itemize}
    \item \textbf{Horizontal Major Axis:}  
    \[
    \frac{(x-h)^2}{a^2} + \frac{(y-k)^2}{b^2} = 1
    \]
    where:
    \begin{itemize}
        \item \( (h, k) \) is the center,
        \item \( a > b \) (semi-major axis \( a \), semi-minor axis \( b \)),
        \item \( c^2 = a^2 - b^2 \), where \( c \) is the focal distance.
    \end{itemize}
    
    \item \textbf{Vertical Major Axis:}  
    \[
    \frac{(y-k)^2}{a^2} + \frac{(x-h)^2}{b^2} = 1
    \]
    with the same parameters as above.
\end{itemize}
\begin{remarkbox}
    \textbf{Properties of Ellipses:}
    \begin{itemize}
        \item \textit{Vertices:} Located \( a \) units from the center along the major axis.
        \item \textit{Foci:} Located \( c \) units from the center along the major axis, where \( c^2 = a^2 - b^2 \).
        \item \textit{Eccentricity:} Defined as \( e = \frac{c}{a} \), with \( 0 < e < 1 \).
    \end{itemize}
    \end{remarkbox}
\end{definitionbox}

\subsection*{Verifying an Ellipse}
\addcontentsline{toc}{subsubsection}{Verifying an Ellipse}
\begin{examplebox} 
Show that the equation  
\[
4x^2 + 9y^2 = 36
\]
represents an ellipse and determine its key features.

\begin{solutionbox}
\begin{itemize}
    \item Rewrite the equation in standard form:
    \[
    \frac{x^2}{9} + \frac{y^2}{4} = 1.
    \]
    \item The ellipse is centered at \( (0, 0) \) with \( a = 3 \), \( b = 2 \), and \( c = \sqrt{a^2 - b^2} = \sqrt{5} \).
    \item The foci are \( (\pm \sqrt{5}, 0) \), and the vertices are \( (\pm 3, 0) \).
\end{itemize}
\end{solutionbox}
\end{examplebox}

\section*{Parabola}
\addcontentsline{toc}{subsection}{Parabola}
\begin{definitionbox}
A \textbf{parabola} is the set of all points in a plane equidistant from a fixed point (the \textit{focus}) and a fixed line (the \textit{directrix}).
\begin{figure}[H]
    \centering
    \includegraphics[width=0.45\textwidth]{parabola.jpg}
    \caption{Diagram of a parabola.}
\end{figure}
\begin{intuitionbox}
    A parabola can be thought of as the trajectory of an object under uniform acceleration, such as the path of a ball thrown in the air.
    \begin{figure}[H]
        \centering
        \includegraphics[width=0.65\textwidth]{parabola_ball_example.png}
        \caption{Parabolic trajectory of a ball.}
    \end{figure}
    \end{intuitionbox}
\end{definitionbox}

\subsection*{Standard Forms of a Parabola}
\addcontentsline{toc}{subsubsection}{Standard Forms of a Parabola}

\begin{definitionbox}
The equation of a parabola depends on whether it opens horizontally or vertically:

\begin{itemize}
    \item \textbf{Opens Right or Left (Horizontal Axis):}  
    \[
    (y-k)^2 = 4p(x-h)
    \]
    \begin{itemize}
        \item \( (h, k) \) is the vertex.
        \item \( p \) is the directed distance from the vertex to the focus.
        \item The focus is at \( (h+p, k) \), and the directrix is the vertical line \( x = h - p \).
    \end{itemize}

    \item \textbf{Opens Up or Down (Vertical Axis):}  
    \[
    (x-h)^2 = 4p(y-k)
    \]
    \begin{itemize}
        \item The vertex and \( p \) are the same as above.
        \item The focus is at \( (h, k+p) \), and the directrix is the horizontal line \( y = k - p \).
    \end{itemize}
\end{itemize}
\begin{remarkbox}
    \textbf{Properties of Parabolas:}
    \begin{itemize}
        \item \textit{Focus:} Located \( p \) units from the vertex along the axis of symmetry.
        \item \textit{Directrix:} A line perpendicular to the axis of symmetry at a distance \( p \) from the vertex.
        \item \textit{Axis of Symmetry:} A line that passes through the focus and is perpendicular to the directrix.
    \end{itemize}
    \end{remarkbox}
\end{definitionbox}

\subsection*{Verifying a Parabola}
\addcontentsline{toc}{subsubsection}{Verifying a Parabola}
\begin{examplebox}  
Show that the equation  
\[
y^2 = 12x
\]
represents a parabola and determine its key features.

\begin{solutionbox}
\begin{itemize}
    \item The equation is in the standard form \( y^2 = 4px \), with \( 4p = 12 \), so \( p = 3 \).
    \item The parabola opens to the right, with vertex \( (0, 0) \), focus \( (3, 0) \), and directrix \( x = -3 \).
\end{itemize}
\end{solutionbox}
\end{examplebox}

\section*{Hyperbola}
\addcontentsline{toc}{subsection}{Hyperbola}
\begin{definitionbox}  
A \textbf{hyperbola} is the set of all points in a plane such that the absolute difference of their distances to two fixed points (called the \textit{foci}) is constant.
\begin{figure}[H]
    \centering
    \includegraphics[width=0.75\textwidth]{hyperbola.jpg}
    \caption{Diagram of a hyperbola.}
\end{figure}
\end{definitionbox}

\begin{intuitionbox}
A hyperbola appears in real-world phenomena such as satellite orbits, radio wave propagation, and the paths of comets.
\begin{figure}[H]
    \centering
    \includegraphics[width=0.5\textwidth]{hyperbola_comparison_to_other_shapes.png}
    \caption{Hyperbolic orbits can have greater eccentricity than parabolic ones.}
\end{figure}
\begin{figure}[H]
    \centering
    \includegraphics[width=0.75\textwidth]{example_light_hyperbola.jpg}
    \caption{ A hyperbolic mirror used to collect light from distant stars.}
\end{figure}
\end{intuitionbox}

\subsection*{Standard Forms of a Hyperbola}
\addcontentsline{toc}{subsubsection}{Standard Forms of a Hyperbola}
\begin{definitionbox}
    A hyperbola is defined by the difference of distances to two fixed points (foci) being constant. Its standard equation depends on the orientation of its transverse axis:  
    \begin{itemize}
        \item \textbf{Horizontal Transverse Axis:}  
        \[
        \frac{(x-h)^2}{a^2} - \frac{(y-k)^2}{b^2} = 1,
        \]
        where \( (h, k) \) is the center, \( a \) is the distance from the center to each vertex, and \( c^2 = a^2 + b^2 \) defines the distance from the center to each focus.  
        
        \item \textbf{Vertical Transverse Axis:}  
        \[
        \frac{(y-k)^2}{a^2} - \frac{(x-h)^2}{b^2} = 1.
        \]
    \end{itemize}
    \begin{remarkbox}
        \textbf{Properties of Hyperbolas:}
        \begin{itemize}
            \item \textit{Foci:} Located \( c \) units from the center along the transverse axis, where \( c^2 = a^2 + b^2 \).
            \item \textit{Asymptotes:} Lines that the hyperbola approaches but never touches, given by:
            \[
            y = k \pm \frac{b}{a}(x-h) \quad \text{(horizontal)}.
            \]
            \item \textit{Vertices:} Located \( a \) units from the center along the transverse axis.
        \end{itemize}
        \end{remarkbox}
\end{definitionbox}

\subsection*{Verifying a Hyperbola}
\addcontentsline{toc}{subsubsection}{Verifying a Hyperbola}
\begin{examplebox}
Show that the equation  
\[
9x^2 - 16y^2 = 144
\]
represents a hyperbola and determine its key features.

\begin{solutionbox}
\begin{itemize}
    \item Rewrite the equation in standard form:
    \[
    \frac{x^2}{16} - \frac{y^2}{9} = 1.
    \]
    \item The hyperbola is centered at \( (0, 0) \) with \( a = 4 \), \( b = 3 \), and \( c = \sqrt{a^2 + b^2} = 5 \).
    \item The vertices are \( (\pm 4, 0) \), the foci are \( (\pm 5, 0) \), and the asymptotes are \( y = \pm \frac{3}{4}x \).
\end{itemize}
\end{solutionbox}
\end{examplebox}

\section*{Eccentricity and Directrix}
\addcontentsline{toc}{subsection}{Eccentricity and Directrix}

\begin{definitionbox}The \textbf{eccentricity} \( e \) of a conic section is defined as the ratio of the distance from any point on the conic to its focus, divided by the perpendicular distance from that point to the nearest directrix. This value is constant for a given conic and determines its type:
\begin{itemize}
    \item If \( e = 1 \), the conic is a \textbf{parabola}.
    \item If \( e < 1 \), the conic is an \textbf{ellipse}.
    \item If \( e > 1 \), the conic is a \textbf{hyperbola}.
\end{itemize}

\begin{remarkbox}
For a \textbf{circle}, the eccentricity is \( e = 0 \).
\end{remarkbox}

The \textbf{directrix} of a conic section is a fixed line that, together with the focus, helps define the conic. 
\begin{itemize}
    \item \textbf{Parabolas} have one focus and one directrix.
    \item \textbf{Ellipses} and \textbf{hyperbolas} (excluding circles) have two foci and two corresponding directrices.
\end{itemize}
\end{definitionbox}

\begin{illustrationbox}
    \begin{figure}[H]
        \centering
        \begin{minipage}{0.45\textwidth}
            \centering
            \includegraphics[width=\linewidth]{eccentricity and directrix.png}
            \caption{Eccentricity and directrix of conic sections.}
        \end{minipage}
        \hfill
        \begin{minipage}{0.45\textwidth}
            \centering
            \includegraphics[width=\linewidth]{hyperbola directrix.png}
            \caption{Directrix of a hyperbola.}
        \end{minipage}
        
        \vspace{1em} % Adds vertical space before the next row
        
        \begin{minipage}{0.6\textwidth}
            \centering
            \includegraphics[width=\linewidth]{ellipse directrix.png}
            \caption{Directrix of an ellipse.}
        \end{minipage}
    \end{figure}    
\end{illustrationbox}

\section*{General Equations of Degree Two}
\addcontentsline{toc}{subsection}{General Equations of Degree Two}

\begin{conceptbox}
A general second-degree equation is written as:
\[
Ax^2 + Bxy + Cy^2 + Dx + Ey + F = 0.
\]
The nature of its graph (a conic section) is determined using the \textbf{discriminant}:
\[
\Delta = 4AC - B^2.
\]
\begin{itemize}
    \item If \( \Delta > 0 \), the conic is an \textbf{ellipse}.
    \item If \( \Delta = 0 \), the conic is a \textbf{parabola}.
    \item If \( \Delta < 0 \), the conic is a \textbf{hyperbola}.
\end{itemize}
\begin{remarkbox}
    If \( B \neq 0 \), the coordinate axes are rotated.

    To determine the rotation angle \( \theta \), use:
    \[
    \cot 2\theta = \frac{A - C}{B}.
    \]
\end{remarkbox}
\end{conceptbox}

\section*{Distinguishing Between Conic Sections}
\addcontentsline{toc}{subsection}{Distinguishing Between Conic Sections}

\begin{tipbox}
To classify a conic section, follow these key steps:
\begin{enumerate}
    \item \textbf{Check the discriminant} \( \Delta = 4AC - B^2 \):
    \begin{itemize}
        \item \( \Delta > 0 \) indicates an \textbf{ellipse}.
        \item \( \Delta = 0 \) indicates a \textbf{parabola}.
        \item \( \Delta < 0 \) indicates a \textbf{hyperbola}.
    \end{itemize}
    \item \textbf{Identify the presence of an \( xy \)-term}:
    \begin{itemize}
        \item If \( B \neq 0 \), the axes are rotated.
    \end{itemize}
    \item \textbf{Analyze the equation form}:
    \begin{itemize}
        \item Ellipses and circles have \textbf{both \( x^2 \) and \( y^2 \) terms} with the same sign.
        \item Hyperbolas have \textbf{both \( x^2 \) and \( y^2 \) terms} with opposite signs.
        \item Parabolas have \textbf{only one squared term} (either \( x^2 \) or \( y^2 \), but not both).
    \end{itemize}
\end{enumerate}
\end{tipbox}

\cleardoublepage
\phantomsection
\input{intolecturecontent}
\addcontentsline{toc}{section}{Lecture Content}
\normalsize

\setcounter{page}{1}

\section*{Review from the Previous Lecture}
\addcontentsline{toc}{subsection}{Review from the Previous Lecture}
\begin{remarkbox}
In the previous lecture, we covered important foundational concepts related to polar coordinates and their derivatives. Here’s a brief summary: 

\begin{itemize}
    \item \textbf{Derivative of \( r = f(\theta) \) in Cartesian Coordinates:}
    \large
    \[
        \dfrac{dy}{dx} = \dfrac{\dfrac{dy}{d\theta}}{\dfrac{dx}{d\theta}} = \dfrac{\dfrac{dr}{d\theta}\sin\theta + r\cos\theta}{\dfrac{dr}{d\theta}\cos\theta - r\sin\theta}
    \]
    \normalsize
    This formula helps us compute the slope of the tangent line for polar curves when converted to Cartesian coordinates. 

    \item \textbf{Equation of a Circle:}
    \[
        (x-h)^2 + (y-k)^2 = r^2
    \]
    Here:
    \begin{itemize}
        \item[\labelitemi] \( r \): Radius of the circle
        \item[\labelitemi] \( (h, k) \): Centre of the circle
    \end{itemize}    
\end{itemize}

\begin{notebox}
\textbf{Reminder:} Term Test 1 is scheduled for \textbf{Thursday, January 30th, 2025 (Week 4)}. Make sure to review polar derivatives, transformations, and conic sections!
\end{notebox}
\end{remarkbox}

\section*{Exploring Common Curve Shapes}
\addcontentsline{toc}{subsection}{Exploring Common Curve Shapes}

\subsection*{Parabola}
\addcontentsline{toc}{subsubsection}{Parabola}

\begin{definitionbox}
A \textbf{parabola} is a symmetric curve defined by the quadratic equation:  
\[
    y = ax^2 + bx + c, \quad a \neq 0
\]
To rewrite this equation in vertex form, we complete the square:  
\[
    y = A(x - B)^2 + C
\]

Here:  
\begin{itemize}
    \item \( A \): Determines the direction and "width" of the parabola.  
    \[
        A > 0 \implies \text{The parabola opens upwards.}
    \]  
    \[
        A < 0 \implies \text{The parabola opens downwards.}
    \]
    \item \( (B, C) \): Represents the vertex of the parabola.
    \begin{itemize} 
        \item[\labelitemi] \( B \): Horizontal position of the vertex.  
        \item[\labelitemi] \( C \): Vertical position of the vertex.
    \end{itemize} 
\end{itemize}

\begin{algorithmbox}

    \textbf{Vertex Formula:}  
    To find the vertex when given the standard form \( y = ax^2 + bx + c \), use the formulas:  
    \[
        B = -\frac{b}{2a}, \quad C = f(B)
    \]
    where \( f(B) \) is the value of the quadratic function evaluated at \( x = B \).
\end{algorithmbox}
\end{definitionbox}

\subsection*{Sketching the Region of a Set Defined by a Parabola}
\addcontentsline{toc}{subsubsection}{Sketching the Region of a Set Defined by a Parabola}
\begin{examplebox}
Sketch the region of the set defined by
\[
    R = \{ (x, y) \mid y \geq x^2 + 1 \}.
\]

\begin{remarkbox}
    To sketch the region defined by \( y \geq x^2 + 1 \), we first consider the graph of the parabola \( y = x^2 + 1 \):
    
    \begin{figure}[H]
        \centering
        \includegraphics[width=0.4\textwidth]{x^2 + 1.png}
        \caption{Graph of \( y = x^2 + 1 \).}
        \label{fig:parabola_graph}
    \end{figure}
    
    Next, let’s test some sample points to determine whether they lie in the region \( y \geq x^2 + 1 \):  
    
    \begin{itemize}
        \item For the point \( (-2, 0) \):  
        \[
        y \geq x^2 + 1 \implies 0 \geq (-2)^2 + 1 \implies 0 \geq 5,
        \]
        which is \textbf{false}. Therefore, \( (-2, 0) \) is not in the region.
        
        \item For the point \( (0, 2) \):  
        \[
        y \geq x^2 + 1 \implies 2 \geq 0^2 + 1 \implies 2 \geq 1,
        \]
        which is \textbf{true}. Therefore, \( (0, 2) \) is in the region.
    \end{itemize}
\end{remarkbox}

\textit{...cont'd...}
\end{examplebox}

\begin{examplebox}
\textit{...cont'd...}
\begin{solutionbox}
    The region defined by \( y \geq x^2 + 1 \) is shown below:

    \begin{blankbox}
        \begin{figure}[H]
            \centering
            \includegraphics[width=0.4\textwidth]{y = x^2 + 1 shaded.png}
            \caption{Shaded region satisfying \( y \geq x^2 + 1 \).}
            \label{fig:region}
        \end{figure}
    \end{blankbox}
    
    \textbf{How to Determine the Region:}
    \begin{conceptbox}
    To determine the region for \( y \geq x^2 + 1 \):
    \begin{itemize}
        \item The parabola \( y = x^2 + 1 \) acts as a boundary. The inequality \( y \geq x^2 + 1 \) indicates that the region lies above or on this parabola.
        \item The graph of \( y = x^2 + 1 \) opens upwards, so the region \( R \) is the area above this curve, including the curve itself.
        \item The boundary curve \( y = x^2 + 1 \) is part of the region because the inequality includes equality (\( \geq \)).
    \end{itemize}
    \end{conceptbox}
\end{solutionbox}    
\end{examplebox}

\subsection*{Ellipse}
\addcontentsline{toc}{subsubsection}{Ellipse}
\begin{definitionbox}
The equation of an ellipse is defined by
\[
    \dfrac{(x - h)^2}{a^2} + \dfrac{(y - k)^2}{b^2} = 1 \text{.}
\]
\begin{remarkbox}
Recall the equation of the circle, which is based on the equation of the ellipse when \( a = b = 1 \):
\[
    \text{Circle:} \quad (x - h)^2 + (y - k)^2 = r^2 \text{,}
\]
where \( (h, k) \) is the centre, \( a \) represents the \( x \)-axis radius, and \( b \) represents the \( y \)-axis radius.
\end{remarkbox}
\end{definitionbox}

\subsection*{Sketching the Region of a Set Defined by an Ellipse}
\addcontentsline{toc}{subsubsection}{Sketching the Region of a Set Defined by an Ellipse}

\begin{examplebox}
Sketch the region of the set defined by
\[
    A = \{ (x, y) \mid x^2 + 4y^2 > 4 \}.
\]

\begin{remarkbox}
    To sketch the region defined by \( x^2 + 4y^2 > 4 \), we first consider the graph of the ellipse \( x^2 + 4y^2 = 4 \):

    \begin{figure}[H]
        \centering
        \includegraphics[width=0.65\textwidth]{graph of x^2 + 4y^2 = 4.png}
        \caption{Graph of \( x^2 + 4y^2 = 4 \).}
        \label{fig:ellipse_graph}
    \end{figure}

    Next, let’s test some sample points to determine whether they lie in the region \( x^2 + 4y^2 > 4 \):

    \begin{itemize}
        \item For the point \( (0, 0) \):
        \[
        x^2 + 4y^2 > 4 \implies 0^2 + 4(0)^2 > 4 \implies 0 > 4,
        \]
        which is \textbf{false}. Therefore, \( (0, 0) \) is not in the region.

        \item For the point \( (3, 0) \):
        \[
        x^2 + 4y^2 > 4 \implies 3^2 + 4(0)^2 > 4 \implies 9 > 4,
        \]
        which is \textbf{true}. Therefore, \( (3, 0) \) is in the region.
    \end{itemize}
\end{remarkbox}

\textit{...cont'd...}
\end{examplebox}

\begin{examplebox}
\textit{...cont'd...}
\begin{solutionbox}
    The region defined by \( x^2 + 4y^2 > 4 \) is shown below:

    \begin{blankbox}
    \begin{figure}[H]
        \centering
        \includegraphics[width=0.55\textwidth]{x^2 + 4y^2 gt 4 region.png}
        \caption{Shaded region satisfying \( x^2 + 4y^2 > 4 \).}
        \label{fig:ellipse_region}
    \end{figure}
    \end{blankbox}

    \textbf{How to Determine the Region:}
    \begin{conceptbox}
    To determine the region for \( x^2 + 4y^2 > 4 \):
    \begin{itemize}
        \item The ellipse \( x^2 + 4y^2 = 4 \) acts as a boundary. The inequality \( x^2 + 4y^2 > 4 \) indicates that the region lies outside this ellipse.
        \item The equation can be rewritten as \( \frac{x^2}{4} + \frac{y^2}{1} = 1 \), showing that it is an ellipse centered at \( (0,0) \) with a semi-major axis of 2 (along \( x \)-axis) and a semi-minor axis of 1 (along \( y \)-axis).
        \item The boundary curve \( x^2 + 4y^2 = 4 \) is \textbf{not} part of the region because the inequality is strict (\( > \)).
        \item A dashed boundary is used in the sketch to indicate that the ellipse itself is not included in the region.
    \end{itemize}
    \end{conceptbox}
\end{solutionbox}    
\end{examplebox}

\subsection*{Hyperbola}
\addcontentsline{toc}{subsubsection}{Hyperbola}

\begin{definitionbox}
The equation of a hyperbola is defined by
\[
    \dfrac{x^2}{a^2} - \dfrac{y^2}{b^2} = 1
\]

\begin{figure}[H]
    \centering
    \includegraphics[width=0.35\textwidth]{hyperbola horizontal transverse axis.png}
    \caption{Graph of the hyperbola with a horizontal transverse axis.}
    \label{fig:horizontal_hyperbola}
\end{figure}

\[
    \dfrac{y^2}{b^2} - \dfrac{x^2}{a^2} = 1
\]

\begin{figure}[H]
    \centering
    \includegraphics[width=0.35\textwidth]{hyperbola vertical transverse axis.png}
    \caption{Graph of the hyperbola with a vertical transverse axis.}
    \label{fig:vertical_hyperbola}
\end{figure}

\end{definitionbox}

\subsection*{Sketching the Region of a Set Defined by a Hyperbola}
\addcontentsline{toc}{subsubsection}{Sketching the Region of a Set Defined by a Hyperbola}

\begin{examplebox}
Sketch the region of the set defined by
\[
    H = \{ (x, y) \mid \dfrac{x^2}{4} - \dfrac{y^2}{1} > 1 \}.
\]

\begin{remarkbox}
    To sketch the region defined by \( \dfrac{x^2}{4} - \dfrac{y^2}{1} > 1 \), we first consider the graph of the hyperbola \( \dfrac{x^2}{4} - \dfrac{y^2}{1} = 1 \):

    \begin{figure}[H]
        \centering
        \includegraphics[width=0.5\textwidth]{x^2 over 4 + y^2 = 1.png}
        \caption{Graph of \( \dfrac{x^2}{4} - \dfrac{y^2}{1} = 1 \).}
        \label{fig:hyperbola_graph}
    \end{figure}

    Next, let’s test some sample points to determine whether they lie in the region \( \dfrac{x^2}{4} - \dfrac{y^2}{1} > 1 \):

    \begin{itemize}
        \item For the point \( (0, 0) \):
        \[
        \dfrac{x^2}{4} - \dfrac{y^2}{1} > 1 \implies \dfrac{0^2}{4} - \dfrac{0^2}{1} > 1 \implies 0 > 1,
        \]
        which is \textbf{false}. Therefore, \( (0,0) \) is not in the region.

        \item For the point \( (3, 0) \):
        \[
        \dfrac{x^2}{4} - \dfrac{y^2}{1} > 1 \implies \dfrac{3^2}{4} - \dfrac{0^2}{1} > 1 \implies \dfrac{9}{4} > 1,
        \]
        which is \textbf{true}. Therefore, \( (3,0) \) is in the region.
    \end{itemize}
\end{remarkbox}

\textit{...cont'd...}
\end{examplebox}

\begin{examplebox}
\textit{...cont'd...}
\begin{solutionbox}
    The region defined by \( \dfrac{x^2}{4} - \dfrac{y^2}{1} > 1 \) is shown below:
    
    \begin{blankbox}
    \begin{figure}[H]
        \centering
        \includegraphics[width=0.75\textwidth]{x^2 over 4 + y^2 gt 1.png}
        \caption{Shaded region satisfying \( \dfrac{x^2}{4} - \dfrac{y^2}{1} > 1 \).}
        \label{fig:hyperbola_region}
    \end{figure}
    \end{blankbox}

    \textbf{How to Determine the Region:}
    \begin{conceptbox}
    To determine the region for \( \dfrac{x^2}{4} - \dfrac{y^2}{1} > 1 \):
    \begin{itemize}
        \item The hyperbola \( \dfrac{x^2}{4} - \dfrac{y^2}{1} = 1 \) acts as a boundary. The inequality \( > 1 \) indicates that the region lies outside the branches of the hyperbola.
        \item The equation shows that the hyperbola has a center at \( (0,0) \), transverse axis along the \( x \)-axis, and asymptotes \( y = \pm \frac{x}{2} \).
        \item The boundary curve \( \dfrac{x^2}{4} - \dfrac{y^2}{1} = 1 \) is \textbf{not} part of the region because the inequality is strict (\( > \)).
        \item A dashed boundary is used in the sketch to indicate that the hyperbola itself is not included in the region.
    \end{itemize}
    \end{conceptbox}
\end{solutionbox}    
\end{examplebox}

\section*{Section 2.1/2.2: Welcome to Linear Algebra...}
\addcontentsline{toc}{subsection}{Welcome to Linear Algebra...}

\subsection*{Well... not really!}
\addcontentsline{toc}{subsubsection}{Well... not really!}

Welcome to MAT232! While the name might suggest a course in linear algebra, this course remains focused on multivariable calculus. However, linear algebra concepts will be integrated into our discussions, particularly when we explore vectors and their applications.

\subsection*{Review of Cartesian Coordinates in Two Dimensions}
\addcontentsline{toc}{subsubsection}{Review of Cartesian Coordinates in Two Dimensions}

\begin{remarkbox}
Before expanding into three dimensions, let's recall the familiar Cartesian coordinate system in \( \mathbb{R}^2 \), where every point is represented as an ordered pair \( (x, y) \) on the \( xy \)-plane.

\begin{blankbox}
\begin{figure}[H]
    \centering
    \includegraphics[width=0.35\textwidth]{cartesian plane in 2D.png}
    \caption{The Cartesian coordinate plane in \( \mathbb{R}^2 \).}
    \label{fig:2d_cartesian}
\end{figure}
\end{blankbox}
\end{remarkbox}

\subsection*{Introducing Three-Dimensional Cartesian Coordinates}
\addcontentsline{toc}{subsubsection}{Introducing Three-Dimensional Cartesian Coordinates}

\begin{conceptbox}
Now, we step into the three-dimensional space, \( \mathbb{R}^3 \), by introducing a third coordinate, \( z \). Each point in \( \mathbb{R}^3 \) is now represented as an ordered triple \( (x, y, z) \). The additional \( z \)-axis extends perpendicular to the \( xy \)-plane, allowing for depth perception in our coordinate system.

\begin{blankbox}
    \begin{figure}[H]
        \centering
        \includegraphics[width=0.5\textwidth]{cartesian plane in 3D.png}
        \caption{The Cartesian coordinate system in \( \mathbb{R}^3 \), including the \( z \)-axis.}
        \label{fig:3d_cartesian}
    \end{figure}
\end{blankbox}

Understanding this extension is crucial for working with vectors, planes, and other geometric structures in higher dimensions.
\end{conceptbox}

\begin{notebox}
\underline{In 2D:} \\
Notice that \( \mathbb{R}^2 = \mathbb{R} \times \mathbb{R} \), where the first \( \mathbb{R} \) represents the \( x \)-values and the second \( \mathbb{R} \) represents the \( y \)-values.
\\
\underline{Now, in 3D:} \\
Notice that \( \mathbb{R}^3 = \mathbb{R} \times \mathbb{R} \times \mathbb{R} \).
\begin{itemize}
    \item The first \( \mathbb{R} \) represents the \( x \)-values;
    \item The second \( \mathbb{R} \) represents the \( y \)-values;
    \item The third \( \mathbb{R} \) represents the \( z \)-values.
\end{itemize}
\end{notebox}

\section*{Example of Plotting in a 3D Cartesian Plane}
\addcontentsline{toc}{subsection}{Example of Plotting in a 3D Cartesian Plane}
\begin{examplebox}
Plot the points \( (-1, 2, -3) \) and \( (2, -4, 2) \).
\begin{illustrationbox}
    \begin{blankbox}
    \begin{figure}[H]
        \centering
        \begin{minipage}{0.35\textwidth}
            \centering
            \includegraphics[width=\linewidth]{plotting the 2 points.png}
            \caption{Illustration from Desmos.}
            \label{fig:3d_point1}
        \end{minipage}
        \hfill
        \begin{minipage}{0.6\textwidth}
            \centering
            \includegraphics[width=\linewidth]{plotting the 2 points (from lecture).png}
            \caption{Illustration from lecture.}
            \label{fig:3d_point2}
        \end{minipage}
    \end{figure}
    \end{blankbox}
\end{illustrationbox}
\begin{conceptbox}
To plot a point in 3D space, locate the corresponding \( x \), \( y \), and \( z \) values on the axes. The point is then represented by the intersection of the three coordinate planes.
\end{conceptbox}
\begin{tipbox}
    Trace the path from the origin to the point to visualize its position in 3D space. This approach helps in understanding the spatial relationships between points.
\end{tipbox}
\end{examplebox}

\section*{Understanding Planes in 3D}
\addcontentsline{toc}{subsection}{Understanding Planes in 3D}

\begin{conceptbox}
In a 2D world, there is no notion of height when considering the \( xy \)-plane. However, in a 3D world, we introduce the \( z \)-coordinate. 

Here are the fundamental planes in a 3D Cartesian coordinate system:

\begin{blankbox}
\subsubsection*{The \( xy \)-Plane (\( z = 0 \))}
\[
    (x, y, 0)
\]
\begin{figure}[H]
    \centering
    \includegraphics[width=0.65\textwidth]{xy plane.png}
    \caption{The \( xy \)-plane where \( z = 0 \).}
    \label{fig:xy_plane}
\end{figure}
\end{blankbox}

\textit{...cont'd...}
\end{conceptbox}

\begin{conceptbox}
\textit{...cont'd...}

\begin{blankbox}
    \subsubsection*{The \( yz \)-Plane (\( x = 0 \))}
    \[
        (0, y, z)
    \]
    \begin{figure}[H]
        \centering
        \includegraphics[width=0.4\textwidth]{yz plane.png}
        \caption{The \( yz \)-plane where \( x = 0 \).}
        \label{fig:yz_plane}
    \end{figure}
\end{blankbox}

\begin{blankbox}
    \subsubsection*{The \( xz \)-Plane (\( y = 0 \))}
    \[
        (x, 0, z)
    \]
    \begin{figure}[H]
        \centering
        \includegraphics[width=0.4\textwidth]{xz plane.png}
        \caption{The \( xz \)-plane where \( y = 0 \).}
        \label{fig:xz_plane}
    \end{figure}
    \end{blankbox}
\end{conceptbox}

\section*{Transitioning from 2D to 3D}

\begin{examplebox}{Visualizing a Line in 3D}
Consider the equation \( y = 2 \) on a 2D Cartesian plane:

\begin{figure}[H]
    \centering
    \includegraphics[width=0.35\textwidth]{sample_image.jpg}
    \caption{The line \( y = 2 \) in 2D.}
    \label{fig:line_2d}
\end{figure}

In a 3D space, this extends infinitely along the \( z \)-axis, forming a vertical plane parallel to the \( xz \)-plane:

\begin{figure}[H]
    \centering
    \includegraphics[width=0.35\textwidth]{sample_image.jpg}
    \caption{The same line extended into 3D space.}
    \label{fig:line_3d}
\end{figure}

\end{examplebox}

\begin{examplebox}{Visualizing a Circle in 3D}
Consider the equation of a circle:

\[
    x^2 + y^2 = 4
\]

\begin{figure}[H]
    \centering
    \includegraphics[width=0.35\textwidth]{sample_image.jpg}
    \caption{A circle in 2D defined by \( x^2 + y^2 = 4 \).}
    \label{fig:circle_2d}
\end{figure}

If we extend this into the third dimension by allowing any \( z \)-value, it forms a **cylinder**, where the circle acts as the cross-section:

\begin{figure}[H]
    \centering
    \includegraphics[width=0.35\textwidth]{sample_image.jpg}
    \caption{The circle extended into 3D space, forming a cylinder.}
    \label{fig:cylinder_3d}
\end{figure}

\end{examplebox}

\subsection*{Next Lecture: We Discuss Vectors!}

\section*{Lecture Title}
\begin{notebox}
This template is designed for MAT232 lecture notes. Replace this content with your specific lecture details.
\end{notebox}

\section*{Key Concepts}
\begin{definitionbox}
A \textbf{parametric equation} is a set of equations that express the coordinates of the points of a curve as functions of a variable, called a parameter.
\end{definitionbox}

\section*{Examples}
\begin{examplebox}
\textbf{Example 1:} Consider the parametric equations:
\[ x = t, \quad y = t^2, \quad t \in \mathbb{R}. \]
\begin{itemize}
    \item At $t = 0$, $(x, y) = (0, 0)$.
    \item At $t = 1$, $(x, y) = (1, 1)$.
\end{itemize}
This describes a parabola.

\begin{figure}[H]
    \centering
    \includegraphics[width=0.35\textwidth]{sample_image.jpg}
    \caption{Sample image illustrating the concept.}
    \label{fig:sample_image}
\end{figure}

\end{examplebox}

\section*{Theorems and Proofs}
\begin{theorembox}
\textbf{Theorem:} If $x(t)$ and $y(t)$ are differentiable functions, the slope of the curve is given by:
\[ \frac{dy}{dx} = \frac{\frac{dy}{dt}}{\frac{dx}{dt}}, \quad \text{provided } \frac{dx}{dt} \neq 0. \]

\begin{figure}[H]
    \centering
    \includegraphics[width=0.35\textwidth]{sample_image1.jpg}
    \caption{Graphical representation of the theorem.}
    \label{fig:sample_image1}
\end{figure}

\end{theorembox}

\section*{Additional Notes}
\begin{notebox}
Always check the domain of the parameter $t$ when solving problems involving parametric equations.
\end{notebox}

\end{document}
